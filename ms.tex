%:
%:
%%% TO DO 
% 1. Fig.5. add some statistics (skewness, kurtosis, etc) to test whether Gaussian. Add ideal N(mu, sigma) to the panels
% 2. Add a constant to C_ell (trial and error) to see if log(C_ell+cte) is perfectly Gaussian, zero skewness and kurtosis, etc, higher moments
% 3. Mock test in the presence of systematics: (include applying the most rigorous method on both the mocks and data)
%Check the impact of n(z -> 0) on C_ell and fNL constraints
%Try the main selection of LRGs
%Increase the max ell for the cross power spectrum used in χ2
%Test theory code with non-zero fNL mocks
%Test covariance: DR9 fNL constraints using the fNL=100 covariance
% You can check if the shifts in fNL are much different when it applied to mocks without any contamination as well.
%It would also be interesting to take a close look at the weight map with and without the stellar density map, in particular for DECaLS SGC, which is where the effect appears to be quite strong.
%Use Hui’s weights
%Use Rongpu’s weights
% look into analytical covariance, e.g., Gaussian
% Mehdi,
%:
%In order to avoid the confirmation bias, we should not select the model with nstar as our best model, since we don’t have other indicators showing this model is more reasonable than the current default.
%On the other hand, we don’t believe fnl is off by 30 from zero, trusting that the Planck cannot be that wrong. Then, we can assume the true cosmology based on the Planck and consider our analysis as an exercise to identify major known/unknown(?) systematics, so that we can be ready for the upcoming spectroscopic fnl. (edited) 
%
%
%Mehdi Rezaie
%  < 1 minute ago
%Sounds like a good argument — we cannot justify using nStar based on the mean density and cross spectra chi-2 tests.
%One other takeaway would be that our result does indeed motivate further investments and studies of the local stellar density and MW extinction if one wants to achieve a robust σ(fNL)~5 with DESI (or ~1 with LSST+CMB).
%
%In the SV3 LRG selection, the TS density is ~800 deg^-2 (in the main selection, the LRG density is ~600 deg^-2). I think some additional cuts, by removing a few percents, we can improve the purity of the selection. It’s worth talking about that with Rongpu. As we have the spectro catalog, I think that it is clearly possible. 
%
%Christophe. 

\documentclass[fleqn, usenatbib]{mnras}
\usepackage{style} % the settings are all here 
\usepackage{booktabs,array}  

\title[PNG with DESI Imaging]{Local primordial non-Gaussianity from the large-scale clustering of photometric DESI luminous red galaxies}

% If you need two or more lines of authors, add an extra line using \newauthor

\author[M. Rezaie et al.]{Mehdi Rezaie$^{1}$, Ashley J. Ross$^{2}$, Hee-Jong Seo$^{3}$,  Hui Kong$^{2}$,  Anna Porredon$^{2}$,\newauthor
Lado Samushia$^{1}$, Edmond Chaussidon$^{4}$, Rongpu Zhou$^{5}$, Alex Krolewski$^{6,7,8}$,\newauthor
Arnaud de Mattia$^{4}$,  Santi Avila$^{?}$, Benedict Bahr-Kalus$^{9}$, Jose Bermejo-Climent$^{?}$,\newauthor Florian Beutler$^{10}$, Klaus Honscheid$^{2,11}$, Eva Mueller$^{?}$, Adam Myers$^{?}$,\newauthor
Nathalie Palanque-Delabrouille$^{4,5}$, Will Percival$^{6,7,8,?}$, Christophe Yeche$^{4}$
\\~\\
% List of institutions
$^{1}$Department of Physics, Kansas State University, 116 Cardwell Hall, Manhattan, KS 66506, USA\\
$^{2}$Center for Cosmology and AstroParticle Physics, The Ohio State University, 191 West Woodruff Avenue, Columbus, OH 43210, USA\\
$^{3}$Department of Physics and Astronomy, Ohio University, Athens, OH 45701, USA\\
$^{4}$IRFU, CEA, Universite Paris-Saclay, F-91191 Gif-sur-Yvette, France\\
$^{5}$Lawrence Berkeley National Laboratory, 1 Cyclotron Road, Berkeley, CA 94720, USA\\
$^{6}$Department of Physics and Astronomy, University of Waterloo, 200 University Ave W, Waterloo, ON N2L 3G1, Canada\\
$^{7}$Perimeter Institute for Theoretical Physics, 31 Caroline St. North, Waterloo, ON N2L 2Y5, Canada\\
$^{8}$Waterloo Centre for Astrophysics, University of Waterloo, 200 University Ave W, Waterloo, ON N2L 3G1, Canada\\
$^{9}$Korea Astronomy and Space Science Institute, Yuseong-gu, Daedeok-daero 776, Daejeon 34055, Republic of Korea\\
$^{10}$Institute for Astronomy, University of Edinburgh, Royal Observatory, Blackford Hill, Edinburgh EH9 3HJ, UK\\
$^{11}$Department of Physics, The Ohio State University, 191 West Woodruff Avenue, Columbus, OH 43210, USA
}

% Don't change these lines
\begin{document}
\label{firstpage}
\pagerange{\pageref{firstpage}--\pageref{lastpage}}
%:
\maketitle


\begin{abstract}
This paper uses the angular power spectrum of luminous red galaxies (LRGs) selected from the Dark Energy Spectroscopic Instrument (DESI) imaging surveys to constrain the local primordial non-Gaussianity parameter $\fnl$. Our sample comprises over $12$ million LRG targets, spanning approximately $14,000$ square degrees of the sky, with redshifts ranging from $0.2< z < 1.35$. Galactic extinction, survey depth, and astronomical seeing are identified as the primary sources of systematic error using feature selection and cross-correlation techniques. Linear regression and artificial neural networks are applied to mitigate systematics and alleviate excess clustering signals on large scales. Our treatment methods are tested and calibrated rigorously against log-normal density simulations with and without $\fnl$ and systematic effects. We find that the neural network treatment outperforms linear regression in reducing remaining systematics in the DESI LRG sample. Assuming the universality relation, we find $\fnl = 47^{+14~(+29)}_{-11~(-22)}$ at $68\%$~($95\%$) confidence. Applying a more aggressive systematics treatment that includes regression against the full set of imaging maps we identified, our maximum likelihood value changes only slightly to $\fnl \sim 50$, but the uncertainty on $\fnl$ increases due to the aggressive treatment removing large-scale clustering information. We apply a series of robustness tests (e.g., cuts on imaging, declination, or scales) that show remarkable consistency in the obtained constraints. Our fiducial result can either be interpreted as a strong detection of non-zero $\fnl$ with the probability $P(\fnl>0)=99.9$ per cent, inconsistent with what is measured from Planck, or evidence for unknown sources of variation in the observed galaxy density (e.g., from calibration errors in photometric zero-point determination or in Galactic extinction corrections). Our results motivate follow-up studies of $\fnl$ with DESI spectroscopic samples, where the inclusion of 3D clustering modes should help separate imaging systematics.
\end{abstract}

% Select between one and six entries from the list of approved keywords.
% Don't make up new ones.
\begin{keywords}
cosmology: inflation - large-scale structure of the Universe
\end{keywords}


%%%%%%%%%%%%%%%%% BODY OF PAPER 
\section{Introduction}
\label{sec:introduction}
Current observations of the cosmic microwave background (CMB), large-scale structure (LSS), and supernovae (SN) are explained by a cosmological model that consists of dark energy, dark matter, and ordinary luminous matter, which has gone through a phase of rapid expansion, known as \textit{inflation},  at its early stages \citep[see, e.g.,][]{weinberg2013observational}. The theory of inflation elegantly addresses fundamental issues with the hot Big Bang theory, such as the isotropy of the CMB temperature, absence of magnetic monopole, and flatness of the Universe. At the end of inflation, the Universe was reheated and primordial fluctuations are generated to seed the subsequent growth of structure. While the presence of an inflationary era is certain but the details of the inflation field still remain highly unknown, and statistical properties of primordial fluctuations pose as one of the puzzling questions in modern observational cosmology. Analyses of cosmological data have revealed that initial conditions of the Universe are consistent with Gaussian fluctuations; however, there are some classes of models that predict some levels of non-Gaussianities in the primordial gravitational field. In its simplest form, primordial non-Gaussianity depends on the local value of the gravitational potential $\phi$ and is parameterized by a nonlinear parameter $\fnl$ \citep{komatsu2001acoustic},
\begin{equation}
    \Phi = \phi + \fnl [\phi^{2} -  <\phi^{2}>].
\end{equation}
Standard slow roll inflation predicts $\fnl$ to be of order $10^{-2}$, while multifield theories predict considerably higher values than unity. Therefore, a robust measurement of $\fnl$ can be considered as the first stepping stone toward better understanding the physics of the early Universe. 

Current tightest bound on $\fnl$ comes from the three-point clustering analysis of the CMB temperature anisotropies by the Planck satellite, $\fnl=0.9\pm 5.0$ \citep{akrami2019planck}. CMB S4, next generation of CMB experiments, will improve this constraint but since CMB is limited by cosmic variance, it alone cannot further enhance to break the degeneracy amongst inflationary models. However, combining CMB with LSS data could cancel cosmic variance, partially if not completely, and enhance these limits to a precision level required to differentiate between various inflationary models \citep[see, e.g.,][]{schmittfull2018PhRvD}.  

PNG alters local number density of galaxies by coupling the long and small wavelength modes of dark matter gravitational field, and as a result it introduces a $k^{-2}$-dependent shift in halo bias which leaves its signature on the large scales in the two-point clustering of large-scale structure \citep[see, e.g.,][]{dalal2008imprints}. Measuring $\fnl$ using the scale-dependent bias is however very challenging due to the presence of systematic effects which cause excess clustering signal on the same scales sensitive to $\fnl$. These systematics can be broadly classified into theoretical and observational. Major theoretical systematic effects are caused by the geometry of survey -- a fact that we never observe the full night sky -- which results in coupling different angular modes. The other effect is commonly referred to as integral constraint and is raised due to our estimation of the mean density directly from data itself, which pushes the clustering signal on modes near the size of survey to zero \mr{(Peacock and Nicholson 1991, Wilson et al 2015)}. Ignoring any of these effects leads to biased $\fnl$ constraints \mr{(see, e.g., Riquelme et al 2022)}. On the other hand, observational systematics are primarily caused by varying imaging properties across the sky or calibration issues which leave spurious fluctuations in target density field \mr{(see, e.g., Huterer et al 2013)}. This type of systematic error is much more difficult to handle and has hindered previous studies of local PNG with galaxy and quasar clustering \citep[see, e.g.,][]{Ho2015JCAP...05..040H}. For instance, \cite{pullen2013systematic} found that the level of systematic contamination in the quasar sample of SDSS \mr{DRX} does not allow a robust $\fnl$ measurement. These imaging systematic issues are expected to be severe for wide-area galaxy surveys that observe the night sky closer to the Galactic plane and attempt to loosen the selection criteria to incorporate fainter targets. 


Assuming imaging systematics are under control, the next generation of galaxy surveys such as DESI and the Rubin Observatory are forecast to yield unprecedented constraints on $\fnl$. Therefore, one of the primary focus of this work is to present an exquisite study of imaging systematic error and enhanced statistical tools to address the data quality for measuring $\fnl$. In this paper, we use the photometric sample of galaxies from the DESI Legacy Imaging Surveys Data Release 9, hereafter referred to as DR9, to constrain the local primordial non-Gaussianity parameter $\fnl$, while testing the robustness of our results against various sources of systematic effects. We also make use of spectroscopic data from DESI Survey Validation to determine the redshift distribution of galaxies. We cross correlate the DR9 density field with the templates of imaging realities to assess the effectiveness of treatment methods and to characterize the significance of residual systematic error. Section \ref{sec:data} describes the DR9 sample and simulations with and without PNG and imaging systematic effects, and Section \ref{sec:method} outlines the theory for modeling angular power spectrum and analysis techniques for quantifying various observational systematic effects. Finally, we present the results in Section \ref{sec:results}, and conclude with a comparison to previous $\fnl$ constraints in Section \ref{sec:conclusion}.
\section{Data}
\label{sec:data}
Luminous red galaxies (LRGs) are massive galaxies that occupy massive halos, lack active star formation, and are considered as one of the highly biased tracers of large scale structure. Redshift of LRGs can be easily determined from  a break around 4000 \AA~in their spectra. LRGs are widely targeted in previous galaxy redshift surveys \citep[see, e.g.,][]{eisenstein2001spectroscopic, prakash2016sdss}, and their clustering and redshift properties are well studied \citep[see, e.g.,][]{alam2021completed}. DESI is expected to collect spectra of millions of LRGs covering the redshift range of $0.4<z<1.0$ over the span of its five year mission. Targets for DESI spectroscopy are selected from imaging surveys; The ground-based surveys that probe the sky in the optical bands are the Mayall z-band Legacy Survey using the Mayall telescope at Kitt Peak \citep{dey2018overview}, the Beijing–Arizona Sky Survey using the Bok telescope at Kitt Peak \citep{zou2017project}, and the Dark Energy Camera Legacy Survey on the Blanco 4m telescope \citep[DECaLS][]{flaugher2015dark}. Additionally, the Legacy Surveys program incorporates observations conducted from the same instrument under the Dark Energy Survey \citep{abbott2016dark}, which constitutes for about $1130 \deg^{2}$ of their southern sky footprint. The BASS+MzLS footprint can be distinguished from the DECaLS by applying DEC $> 32.375$ degrees, although there is an overlap between the two region for calibration. 

\subsection{DESI Imaging DR9 LRGs}
We use photometric LRGs selected from the DESI Imaging Surveys Data Release 9 \citep[DR9;][]{dey2018overview} using the selection designed for the DESI 1\% survey \mr{(CITE)}, described as SV3 in \cite{zhou2022target}. The color-magnitude cuts are described in the $g$, $r$, $z$ bands in the optical and $W1$ band in the infrared, and summarized here in Tab. \ref{tab:ts}. The implementation of these selections in the DESI pipeline is described in \cite{myers2022}. DESI-like LRGs are selected brighter than the survey depth limits, and thus the sample density field is nearly homogenous. To further reduce stellar contamination, the sample is masked for bright stars, foreground bright galaxies as well as clusters of galaxies \footnote{See the maskbits at \url{https://www.legacysurvey.org/dr9/bitmasks/}}. Then, it is binned into \textsc{healpix} \citep{gorski2005healpix} at $\textsc{nside}=256$ to construct the density map with an average density of $800$ deg$^{-2}$ with a coverage around \mr{$14,000$} square degrees of the sky. The density map is corrected for pixel incompleteness in the density field of LRGS using a catalog of random points, hereafter referred to as randoms, uniformly scattered over the footprint with the same cuts and masks applied to the DR9 LRGs. 
\begin{table*}
    \caption{Selection criteria for the DESI-like LRG targets \citep{zhou2022target}.}
    \label{tab:ts}
    \centerline{%
    \begin{tabular}{lll}
    \hline
    \hline
     \textbf{Footprint} & \textbf{Criterion} &\textbf{Description}\\
      \hline
      \hline   
    &  $z_{\rm fiber} < 21.7$  & Faint limit  \\
          DECaLS & $z - W1 > 0.8 \times (r - z) - 0.6$ & Stellar rejection  \\
     & $[(g-r >1.3)~{\rm AND}~((g-r) > -1.55*(r-W1) + 3.13)]~{\rm OR}~(r -W 1 > 1.8)$ & Remove low-z galaxies \\
     & $[(r-W1 > (W1 - 17.26)*1.8)~{\rm AND}~(r - W1 > W1 - 16.36)]~{\rm OR}~(r-W1 > 3.29)$ & Luminosity cut \\ 
    \hline
    & $z_{\rm fiber} < 21.71$  & Faint limit  \\
 BASS+MzLS    & $z - W1 > 0.8 \times (r - z) - 0.6$ & Stellar rejection  \\
    & $[(g-r >1.34)~{\rm AND}~((g-r) > -1.55*(r-W1) + 3.23)]~{\rm OR}~(r -W 1 > 1.8)$ & Remove low-z galaxies \\
    & $[(r-W1 > (W1 - 17.24)*1.83)~{\rm AND}~(r - W1 > W1 - 16.33)]~{\rm OR}~(r-W1 > 3.39)$ & Luminosity cut \\ 
      \hline
      \end{tabular}
      }
\end{table*}

\begin{figure*}
    \centering
    \includegraphics[width=\textwidth]{figures/dr9data.pdf}
    \caption{DESI Imaging Legacy Survey Data Release 9 Luminous Red Galaxies and imaging properties \citep{dey2018overview}. Top: Observed target density field in deg$^{-2}$. Spurious disconnected islands from the DECaLS North footprint at Declination below $-11$ and parts of the DECaLS South with Declination below $-30$ are dropped from the DR9 sample due to potential calibration issues. Bottom: Mollweide projections of DR9 catalog imaging properties in celestial coordinates.}
    \label{fig:ng}
\end{figure*}
%
%\begin{figure*}
%    \centering
%    \includegraphics[width=0.95\textwidth]{figures/hp_features.pdf}
%    \caption{Mollweide projections of imaging properties in celestial coordinates. }
%    \label{fig:xmaps}
%\end{figure*}

\begin{figure}
    \centering
    \includegraphics[width=0.45\textwidth]{figures/nz_lrg.pdf}
    \caption{Redshift distribution and bias evolution of DESI LRGs \citep{zhou2021clustering, zhou2022target}. The redshift distribution is deducted from spectroscopy and the bias model assumes a constant clustering amplitude.}
    \label{fig:nz}
\end{figure}


Fig. \ref{fig:ng} (top) shows observed density field of DR9 LRGs in deg$^{-2}$ before accounting for any potential systematic effects. There are some disconnected islands, hereafter referred to as \textit{spurious islands}, in the DECaLS North region at Declination below $-11$, which are removed from the sample to minimize potential calibration issues. Additionally, parts of the DECaLS South with Declination below $-30$ are cut from the sample, since similar calibration issues might tamper with our analysis. We present  how these data cuts influence our $\fnl$ constraints in Section \ref{sec:results}. Fig. \ref{fig:ng} (bottom) illustrates the redshift distribution of our sample which is inferred from the DESI Survey Validation data \mr{(CITE)}, and the evolution of  galaxy bias for our LRG sample adapted from \cite{zhou2021clustering}, consistent with the assumption of a constant clustering amplitude.

We study the impact of potential sources of systematic error, mapped into \textsc{healpix} at the same \textsc{nside}. Similar to \cite{zhou2022target}, the properties studied in this work are local stellar density constructed from point-like sources with a g-band magnitude in the range $12 \leq g < 17$ from Gaia Data Release 2 \citep[see,][]{gaiadr2, myers2022};  Galactic extinction E[B-V] from \cite{schlegel1998maps}; and other imaging properties include survey depth (galaxy depth in the g, r, and z bands and PSF depth in W1) and seeing in the g, r, and z bands. These maps are produced by making the histograms of randoms (painted with imaging properties) in \textsc{HEALPix} and averaging over randoms in each pixel. Fig. \ref{fig:pcc} shows the Pearson correlation between galaxy density and imaging properties for the imaging surveys in the top panel and the correlation among imaging properties themselves for the full DESI survey in the bottom panel. There is a strong correlation between galaxy density and depth maps and then the second important property seems to be Galactic foregrounds. There is a little correlation between galaxy density and the W1 depth and psfsize properties. We find that there is a large correlation among the imaging properties themselves, especially between the local stellar density and Milky Way extinction; also, the r-band and g-band properties are more correlated with each other than with the z-band. We follow a template based method to derive a set of weights to account for spurious fluctuations by regressing out galaxy counts against a set of imaging maps or templates.  Because of the inner-correlation amongst the maps, a few subsets of maps are considered as well. These subsets are selected to minimize the correlations among the predictors while having maximum correlation with observed density map.
\begin{itemize}
\item Conservative I: Extinction, depth$_{z}$
\item Conservative II: Extinction, depth$_{z}$, psfsize$_{r}$
\item All Maps: Extinction, depth in $grz$ and $W1$, psfsize in $grz$
\end{itemize}
We also investigate whether including external maps for neutral hydrogen column density \mr{(CITE)} and calibration (e.g., in the z band; \textit{CALIBZ}) could shed light on remaining systematic effects. 

Linear and nonlinear models (approximated using a neural network) are applied to assess the potential of nonlinear systematic error. Parameters of the models are fit by optimizing the negative Poisson log likelihood, $= \sum \lambda - \rho \log(\lambda)$, where the summation runs over pixels, $\rho$ is the galaxy density, and $\lambda$ is either a linear or nonliner model for galaxy density given imaging properties \textbf{x} as input, $\lambda(\textbf{x}) = \log (1+e^{f(\textbf{x})})$. For finding the parameters of the linear model, we perform a Monte Carlo Markov Chain (MCMC) search using the \textsc{emcee} package \mr{(CITE)} and for the nonlinear model we use the implementation from \cite{rezaie2021primordial}; specifically, the nonlinear model is an ensemble of 20 neural network models. Each neural network is constructed with three hidden layers and 20 rectifier units on each layer. Rectifier is identity function for positive input and zero for negative, and it introduces nonlinearities in the neural network. For the linear model we use all data for computing the log of posterior during MCMC while for the nonlinear approach we use $60\%$ of data for training, $20\%$ for validation, and $20\%$ for testing; this is to minimize the chance of over-fitting by the nonlinear model. By changing the permutation of training-testing splits, we test the nonlinear model on entire data. The training is performed for up to 70 training epochs using \textsc{Adam} optimizer, which is a variant of gradient descent, and the learning rate is tuned on the validation set to dynamically varying between $0.001$ and $0.1$, to enable learning robust against local minima. The best model is then selected with the lowest prediction error when applied to the validation set. Finally, we apply the ensemble of 20 best fit models to the test set and average over the predictions. 

Fig. \ref{fig:npred} shows the predicted density fields from the linear model using various sets of imaging maps, and the nonlinear prediction is also shown for comparison. While most of the large-scale spurious fluctuations are explained by just the extinction map and depth in the z band, adding the psfsize in the r band seems to add more fine structure to the predicted density map. Using all maps does not add much structure. Comparing linear to nonlinear with the same input maps, we find that the nonlinear approach yields finer structure due to a higher flexibility.  Overall, both models predict higher density near the boundaries where the surveys meet the high extinction regions of Milky Way. These regions are probably contaminated artifacts entering the selection either via the direct stellar contamination or the impact of extinction on colors.

%\begin{figure}
%    \centering
%    \includegraphics[width=0.45\textwidth]{figures/npred.pdf}
%    \caption{Predicted galaxy counts from template regression. Baseline approach uses imaging maps from Zhou et al. (2022): EBV, galaxy depth in rgz, psfdepth in W1, and psfsize in grz. Conservative I uses EBV and galaxy depth in z, and Conservative II uses EBV, galaxy depth in z, and psfsize in r. In all approaches, the models are regressed on BASS+MzLS, DECaLS North, and DECaLS South separately.}
%    \label{fig:npred}
%\end{figure}

\begin{figure}
    \includegraphics[width=0.45\textwidth]{figures/pcc.pdf} 
    \includegraphics[width=0.45\textwidth]{figures/pccx.pdf}     
    \caption{Top: Pearson-r correlation coefficient between galaxy density and imaging properties in the three imaging regions (top) and between imaging properties themselves for the full DESI footprint (bottom). Solid curves represent the range of correlations observed in 100 randomly selected mock realizations.}
    \label{fig:pcc}
\end{figure}


\subsection{Synthetic lognormal density fields}
Lognormal distributions are shown to be appropriate for describing matter density fluctuations on large scales \citep{coles1991}. Unlike N-body simulations, the generation of lognormal density fields is rather quick and enables a computationaly cheap method to create a large number of realizations, validate analysis pipelines, and construct covariance matrices for error estimation. \textsc{FLASK}  \citep[Full-sky Lognormal Astro-fields Simulation Kit;][]{Xavier_2016} is used to generate series of lognormal galaxy density fields with $\fnl=0$ and $76.92$ using $b(z)=1.43/D(z)$. $1000$ realizations are generated for each $\fnl$. The fiducial cosmology to generate the mocks is based on a flat $\Lambda$CDM universe including one massive neutrino with $m_{\nu}=0.06$ eV, and the rest of cosmological parameters are chosen within $68\%$ of the Planck 2018 results \citep{aghanim2020planck},
\begin{equation*}
    h = 0.67,  \Omega_{M}=0.31, \sigma_{8}=0.8, {\rm and}~ n_{s}=0.97.
\end{equation*}
We use the same fiducial cosmology for the analysis of DR9 sample. These parameters are not degenerate with $\fnl$, however the impact of the fiducial cosmology on $\fnl$ constraints is further investigated in Appendix \mr{?}.
\section{Analysis techniques}
\label{sec:method} 
We address imaging systematics in DESI data by performing a separate treatment for each imaging region (e.g., DECaLS North) within the DESI footprint to reduce the impact of systematic effects specific to that region. Once the imaging systematic weights are obtained for each imaging region separately, we combine the data from all regions to compute the power spectrum for the entire DESI footprint to increase the overall statistical power and enable more robust measurements of $\fnl$. We then conduct robustness tests on the combined data to assess the significance of any remaining systematic effects.


\subsection{Power spectrum estimator}
We first construct the density contrast field from the LRG density, $\rho$,
\begin{align}\label{eq:delta}
    \delta_{g} &= \frac{\rho- \overline{\rho}}{\overline{\rho}},
\end{align}
where the mean galaxy density $\overline{\rho}$ is estimated from the entire LRG sample. As a robustness test, we also analyze the power spectrum from each imaging region individually, in which $\overline{\rho}$ is calculated separately for each region. Then, we use the pseudo angular power spectrum estimator \citep{hivon2002master},
\begin{equation}\label{eq:pusedocell}
        \tilde{C}_{\ell} = \frac{1}{2\ell +1} \sum_{m=-\ell}^{\ell} |a_{\ell m}|^{2},
\end{equation}
where the coefficients $a_{\ell m}$ are obtained by decomposing $\delta_{g}$ into spherical harmonics, $Y_{\ell m}$,
\begin{equation}\label{eq:alm}
        a_{\ell m} = \int d\Omega ~ \delta_{g} W Y^{*}_{\ell m},
\end{equation}
where $W$ represents the survey window that is described by the number of randoms normalized to the expected value.

We use the implementation of \texttt{anafast} from the \textsc{HEALPix} package \citep{gorski2005healpix} to do fast harmonic transforms (Equation \ref{eq:alm}) and estimate the pseudo angular power spectrum of the LRG targets and the cross power spectrum between the LRG targets and the imaging systematic maps.

\subsection{Modelling}
The estimator in Equation \ref{eq:pusedocell} yields a biased power spectrum when the survey sky coverage is incomplete. Specifically, the survey mask causes correlations between different harmonic modes \citep{beutler2014clustering,wilson2017rapid}, and the measured clustering power is smoothed on scales near the survey size. An additional potential cause of systematic error arises from the fact that the mean galaxy density used to construct the density contrast field (Equation \ref{eq:delta}) is estimated from the available data, rather than being known a priori. This introduces what is known as an integral constraint effect, which can cause the power spectrum on modes near the size of the survey to be artificially suppressed, effectively pushing it towards zero \citep{peacock1991large,de2019integral}. Since $\fnl$ is highly sensitive to the clustering power on these scales, it is crucial to account for these systematic effects in the model galaxy power spectrum to obtain unbiased $\fnl$ constraints \citep[see, also,][]{riquelme2022primordial}, which we describe below.
  
The other theoretical systematic issues are however subdominant in the angular power spectrum. For instance, relativistic effects generate PNG-like scale-dependent signatures on large scales, which interfere with measuring $\fnl$ with the scale-dependent bias effect using higher order multipoles of the 3D power spectrum \citep{wang2020}. Similarly, matter density fluctuations with wavelengths larger than survey size, known as super-sample modes, modulate the galaxy 3D power spectrum \citep{castorina2020JCAP}. In a similar way, the peculiar motion of the observer can mimic a PNG-like scale-dependent signature through aberration, magnification and the Kaiser-Rocket effect, i.e., a systematic dipolar apparent blue-shifting in the direction of the observer's peculiar motion \citep{2021JCAP...11..027B}.
  
\subsubsection{Angular power spectrum} 
The relationship between the linear matter power spectrum $P(k)$ and the projected angular power spectrum of galaxies is expressed by the following equation:
\begin{equation}\label{eq:cell}
C_{\ell} = \frac{2}{\pi}\int_{0}^{\infty}\frac{dk}{k}k^{3}P(k)|\Delta_{\ell}(k)|^{2} + N_{\rm shot},
\end{equation}
where $N_{\rm shot}$ is a scale-independent shot noise term. The projection kernel $\Delta_{\ell}(k) = \Delta^{\rm g}_{\ell}(k) + \Delta^{\rm RSD}_{\ell}(k) + \Delta^{\mu}_{\ell}(k)$ includes redshift space distortions \mr{and magnification bias,} and determines the contribution of each wavenumber $k$ to the galaxy power spectrum on mode $\ell$. For more details on this estimator, refer to \cite{Padmanabhan2007}. The non-linearities in the matter power spectrum are negligible for the scales of interest \citep[see, e.g.,][]{Ho2015JCAP...05..040H}. For $\ell=40$, $\Delta_{\ell}(k)$ peaks at $k\sim 0.02~ h\text{Mpc}^{-1}$, which is above the non-linear regime. The FFTLog\footnote{\href{https://github.com/xfangcosmo/FFTLog-and-beyond}{github.com/xfangcosmo/FFTLog-and-beyond}} algorithm and its extension as implemented in \cite{fang2020beyond} are employed to calculate the integrals for the projection kernel $\Delta_{\ell}(k)$, which includes the $l^{\rm th}$ order spherical Bessel functions, $ j_{\ell}(kr)$, and its second derivatives,
\begin{align}
    \Delta^{\rm g}_{\ell}(k) &= \int \frac{dr}{r} r (b+\Delta b) D(r) \frac{dN}{dr} j_{\ell}(kr),\\
    \Delta^{\rm RSD}_{\ell}(k) &= - \int \frac{dr}{r} r f(r) D(r) \frac{dN}{dr} j^{\prime\prime}_{\ell}(kr),\\
    \Delta^{\mu}_{\ell}(k) &= - \ell(\ell+1) \int dr D(r) W_{\mu}(z) j_{\ell}(kr),
\end{align}
where $b$ is the linear bias (dashed curve in Figure \ref{fig:nz}), $D$ represents the linear growth factor normalized as $D(z=0)=1$, $f(r)$ is the growth rate, and $dN/dr$ is the redshift distribution of galaxies normalized to unity and described in terms of comoving distance\footnote{$dN/dr = (dN/dz)(dz/dr) \propto (dN/dz)H(z)$} (solid curve in Figure \ref{fig:nz}). \mr{The magnification bias window function $W_{\mu}(z)$ is}
\begin{equation}
W_{\mu}(z) = (5s-2)\frac{3H^{2}_{0}\Omega_{m}(1+z)}{2c^{2}k^{2}} \int_{z}^{\infty} dz^{\prime}\frac{dN}{dz} \frac{r(z^{\prime}) - r(z)}{r(z^{\prime})r(z)},
\end{equation}
\mr{where $\Omega_{m}$ is the matter density, $H_{0}$ is the Hubble constant\footnote{$H_{0}=100~({\rm km}~{\rm s}^{-1})/(h^{-1}{\rm Mpc})$ and $k$ is in unit of $h {\rm Mpc}^{-1}$}, $c$ is the speed of light, and $s$ is the slope of the number count function \citep{2008PhRvD..77b3512L} which quantifies the response of the number density of galaxies to achromatic changes in the brightness. The parameter $s$ is estimated by shifting all magnitudes by an infinitesimal amount and re-running the color-mag selection, with the caveat that for a fiber flux-selected sample, like the DESI LRG targets, the impact of magnification on fiber flux depends on the shape parameters for each morphology type \citep{zhou2023desi}. The parameter $s$ for our sample is found to vary slightly in the different imaging regions: $s=0.951\pm 0.011$ for BASS+MzLS, $s=0.943 \pm 0.007$ for DECaLS North+DECaLS South, and $s=0.945\pm 0.006$ for DESI\footnote{Private communication with Dr. Rongpu Zhou.}. We fix $s$ to the central values in our analysis.} The PNG-induced scale-dependent shift is given by \citep{slosar2008constraints}
\begin{equation}
\Delta b = b_{\phi}(z) \fnl \frac{3 \Omega_{m} H^{2}_{0}}{2 k^{2}T(k)D(z) c^{2}} \frac{g(\infty)}{g(0)},
\label{eq:scaledepbias}
\end{equation}
where $T(k)$ is the transfer function, and $g(\infty)/g(0) \sim 1.3$ with $g(z)\equiv (1+z) D(z)$ is the growth suppression due to non-zero $\Lambda$ because of our normalization of $D$ \citep[see, e.g.,][]{2010JCAP...07..013R, 2019MNRAS.485.4160M}. We assume the universality relation which directly relates $b_\phi$ to $b$ via $b_{\phi} = 2 \delta_{c}(b - p)$ with $\delta_{c}= 1.686$ representing the critical density for spherical collapse \citep{fillmore1984self}. We fix $p=1$ in our analysis and marginalize over b \citep[see, also,][]{slosar2008constraints,2010JCAP...07..013R,2013MNRAS.428.1116R}. 

\begin{figure}
\centering
\includegraphics[width=0.45\textwidth]{model_mock.pdf}
\caption{The mean power spectrum from the $\fnl=0$ mocks (no contamination) and best-fitting theoretical prediction after accounting for the survey geometry and integral constraint effects. Bottom panel shows the residual power spectrum relative to the mean power spectrum. The dark and light shades represent the $68\%$ error on the mean and one realization, respectively. No imaging systematic cleaning is applied to these mocks.}\label{fig:model_mock}
\end{figure}

\subsubsection{Survey geometry and integral constraint}
We employ a technique similar to the one proposed by \cite{chon2004fast} to account for the impact of the survey geometry on the theoretical power spectrum. The ensemble average for the partial sky power spectrum is related to that of the full sky power spectrum via a mode-mode coupling matrix, ${\rm M}_{\ell \ell^{\prime}}$,
\begin{equation}\label{eq:mixm}
    <\tilde{C}_{\ell}> = \sum_{\ell^{\prime}} {\rm M}_{\ell \ell^{\prime}}<C_{\ell^{\prime}}>.
\end{equation}
We convert this convolution in the spherical harmonic space into a multiplication in the correlation function space. Specifically, we first transform the theory power spectrum (Equation \ref{eq:cell}) to the correlation function, $\hat{\omega}^{\rm model}$. Then, we estimate the survey mask correlation function, $\hat{\omega}^{\rm window}$, and obtain the pseudo-power spectrum,
\begin{align}
    \tilde{C}^{\rm model}_{\ell} &= 2\pi \int \hat{\omega}^{\rm model}\hat{\omega}^{\rm window}~P_{\ell}(\cos\theta) d\cos\theta.
\end{align}
\mr{Figure \ref{fig:mask2pf} illustrates ${\hat \omega}^{\rm window}$ for the different masks representing the DESI footprint and its imaging sub-regions. We present a benchmark of our method against the direct mode-mode coupling matrix approach in Appendix \ref{ssec:windowconv}.} 

The integral constraint is another systematic effect which is induced since the mean galaxy density is estimated from the observed galaxy density, and therefore is biased by the limited sky coverage \citep{peacock1991large}. To account for the integral constraint, the survey mask power spectrum is used to introduce a scale-dependent correction factor that needs to be subtracted from the power spectrum as,
\begin{equation}
     \tilde{C}^{\rm model, IC}_{\ell} = \tilde{C}^{\rm model}_{\ell} - \tilde{C}^{\rm model}_{\ell=0} \left(\frac{\tilde{C}^{\rm window}_{\ell}}{\tilde{C}^{\rm window}_{\ell=0}}\right),
\end{equation}
where $\tilde{C}^{\rm window}$ is the survey mask power spectrum, i.e., the spherical harmonic transform of $\hat{\omega}^{\rm window}$.

\begin{figure}
    \centering
    \includegraphics[width=0.45\textwidth]{figures/mask_2pf.pdf}
    \caption{\mr{The survey mask correlation functions for the imaging regions forming the DESI footprint as a function of angular separation. The inset shows the correlations specifically for angles between $100$ and $180$ degrees.}}
    \label{fig:mask2pf}
\end{figure}



The lognormal simulations are used to validate our survey window and integral constraint correction. Figure \ref{fig:model_mock} shows the mean power spectrum of the $\fnl=0$ simulations (dashed) and the best-fitting theory prediction before and after accounting for the survey mask and integral constraint. The simulations are neither contaminated nor mitigated. The light and dark shades represent the 68\% estimated error on the mean and one single realization, respectively. The DESI mask, which covers around $40\%$ of the sky, is applied to the simulations. We find that the survey window effect modulates the clustering power on $\ell < 200$ and the integral constraint alters the clustering power on $\ell < 6$.

\subsection{Parameter estimation}

\begin{figure*}
\centering
\includegraphics[width=0.85\textwidth]{hist_cl.pdf}
\caption{The distribution of the first bin power spectra and its log transformation from the simulations with $\fnl=0$ (left) and $76.9$ (right). The log transformation largely removes the asymmetry in the distributions.}\label{fig:histcell}
\end{figure*}

Our parameter inference uses standard MCMC sampling. A constant clustering amplitude is assumed to determine the redshift evolution of the linear bias of our DESI LRG targets, $b(z) = b/D(z)$, which is supported by the HOD fits to the angular power spectrum \citep{zhou2021clustering}. In MCMC, we allow $\fnl$, $N_{\rm shot}$, and $b$ to vary, while all other cosmological parameters are fixed at the fiducial values (see \S \ref{ssec:mocks}). The galaxy power spectrum is divided into a discrete set of bandpower bins with $\Delta\ell=2$ between $\ell=2$ and $20$ and $\Delta \ell=10$ from $\ell=20$ to $300$. Each clustering mode is weighted by $2\ell+1$ when averaging over the modes in each bin.

The expected large-scale power is highly sensitive to the value of $\fnl$ such that the the amplitude of the covariance for $C_{\ell}$ is influenced by the true value of $\fnl$, see also \cite{2013MNRAS.428.1116R} for a discussion. As illustrated in the top row of Figure \ref{fig:histcell}, we find that the distribution of the power spectrum at the lowest bin, $2\leq \ell < 4$, is highly asymmetric and its standard deviation varies significantly from the simulations with $\fnl=0$ to $76.9$. We can make the covariance matrix less sensitive to $\fnl$ by taking the log transformation of the power spectrum, $\log C_{\ell}$. As shown in the bottom panels in Figure \ref{fig:histcell}, the log transformation reduces the asymmetry and the difference in the standard deviations between the $\fnl=0$ and $76.9$ simulations. Therefore, we minimize the negative log likelihood defined as,
\begin{equation}\label{eq:likelihood}
-2\log \mathcal{L} = (\log \tilde{C}(\Theta)-\log \tilde{C})^{\dagger} \mathbb{C}^{-1} (\log \tilde{C}(\Theta)-\log \tilde{C}),
\end{equation}
where $\Theta$ represents a container for the parameters $\fnl$, $b$, and $N_{\rm shot}$; $\tilde{C}(\Theta)$ is the (binned) expected pseudo-power spectrum; $\tilde{C}$ is the (binned) measured pseudo-power spectrum; and $\mathbb{C}$ is the covariance on $\log\tilde{C}$ constructed from the $\fnl=0$ log-normal simulations. Log-normal simulations have been commonly used and validated to estimate the covariance matrices for galaxy density fields, and non-linear effects are subdominant on the scales of interest to $\fnl$ \citep[see, e.g.,][]{2017MNRAS.466.1444C, 2021MNRAS.508.3125F}. We also test for the robustness of our results against an alternative covariance constructed from the $\fnl=76.9$ mocks. Flat priors are implemented for all parameters: $\fnl \in [-1000, 1000]$, $N_{\rm shot} \in [-0.001, 0.001]$, and $b \in [0, 5]$. 


\subsection{Calibration of over-correction}\label{ssec:calibration}

The template-based mitigation of imaging systematics removes some of the true clustering signal, and \mout{the amount of the removed signal increases as more maps are used for the regression.} \mr{mitigating with more maps should remove more modes and thus both bias the $\fnl$ estimation and the uncertainty.} We calibrate the over-correction effect using the mocks presented in \S \ref{sec:data}. \mr{One of the main advantages of having two sets of mocks with low and high power at low $\ell$ is that it gives a model for mapping the whole posterior and thus learning how the $\fnl$ constraints degrade as the imaging systematic correction gets greater.}  We apply the neural network model to both the $\fnl=0$ and $76.9$ simulations, with and without imaging systematics, using various sets of imaging systematic maps. Specifically, we consider \textit{non-linear three maps}, \textit{non-linear four maps}, and \textit{non-linear nine maps}. Then, we measure the power spectra from the mocks. We fit both the mean power spectrum and each individual power spectrum from the mocks. \mr{Appendix \ref{ssec:contmocks} presents the impact of the non-linear methods on the mock power spectra, and here we summarize the relevant details for the calibration of over-correction.}

Figure \ref{fig:fnlbias} displays a comparison between the estimates of $\fnl$ before and after mitigation for the clean mocks. The best-fitting estimates \mr{from the mean of the mocks} are represented by the solid curves, and the individual spectra results are displayed as the scatter points. The results from fitting the mean power spectrum of the contaminated mocks are also shown via the dashed curves. We find nearly identical results for the biases caused by mitigation, whether or not the mocks have any contamination, which can be seen by observing the solid and dashed curves displayed on Figure \ref{fig:fnlbias} (see, also, Figure \ref{fig:clmocks}, for a comparison of the mean power spectrum). For clarity, the best-fitting estimates for the individual contaminated data are not shown in the figure.

\begin{figure}
\centering
\includegraphics[width=0.45\textwidth]{figures/fnlbias}
\caption{The \textit{No mitigated, clean} vs \textit{mitigated} $\fnl$ values from the $\fnl=0$ and $76.9$ mocks. The solid (dashed) lines represent the best-fitting estimates from fitting the mean power spectrum of the clean (contaminated) mocks. The scatter points show the best-fitting estimates from fitting the individual spectra for the clean mocks.}\label{fig:fnlbias}
\end{figure}


\mr{As summarized in Table \ref{tab:contmocksmcmc}, we find significant shifts in the best-fitting estimates of $\fnl$ from the fits to the mean spectra. Specifically, we obtain $\Delta \fnl=-12$ for non-linear three maps, $-20$ for non-linear three maps, and $-27$ for non-linear nine maps when analyzing the $\fnl=0$ mocks. Whereas, bigger shifts are noticed for $\fnl=76.9$: we find $\Delta \fnl=-23$ for non-linear three maps, $-39$ for non-linear four maps, and $-72$ for non-linear nine maps.}

To calibrate our methods, we fit a linear curve to the $\fnl$ estimates from the mean power spectrum of the mocks, $f_{\rm NL, no~mitigation, clean} = m_{1} f_{\rm NL, mitigated} + m_{2}$. The $m_{1}$ and $m_{2}$ coefficients for non-linear three, four, and nine maps are summarized in Table \ref{tab:debiasparams}. These coefficients represent the impact of the cleaning methods on the likelihood. The uncertainty in $\fnl$ after mitigation increases by $m_{1}-1$. Figure \ref{fig:fnlbias} also shows that the choice of our cleaning method can have significant implications for the accuracy of the measured $\fnl$, and careful consideration should be given to the selection of the primary imaging systematic maps and the calibration of the cleaning algorithms in order to minimize systematic uncertainties.


\begin{table}
\begin{center}
\caption{Linear parameters employed to de-bias the $\fnl$ constraints to account for the over-correction issue.}\label{tab:debiasparams}
\begin{tabular}{lcc}
\hline
\hline
\textbf{Cleaning Method} & $m_{1}$ & $m_{2}$ \\
\hline
Nonlinear Three Maps & 1.17 & 13.95 \\
Nonlinear Four Maps & 1.32 & 26.97 \\
Nonlinear Nine Maps & 2.35 & 63.5\\
\hline
\end{tabular}
\end{center}
\end{table}

\section{Results}\label{sec:results}
\mr{This section presents our $\fnl$ constraints from the DESI LRG targets. The analysis is not carried out blindly. However, the cleaning methods are decided only based on the cross power spectrum and mean density contrast statistics.} 

\begin{figure*}
    \centering
    \includegraphics[width=0.9\textwidth]{figures/model_dr9.pdf} 
    \caption{The angular power spectrum of the DR9 LRG sample before (\textit{No weight}) and after correcting for imaging systematics using the linear and nonlinear methods with their corresponding best fit theory curves. The solid curve and grey shade respectively represent the mean power spectrum and $68\%$ error from the $\fnl=0$ mocks.}
    \label{fig:cl_dr9}
\end{figure*}

\subsection{DESI imaging LRG sample}
Figure \ref{fig:cl_dr9} shows the measured power spectrum of the DR9 LRG sample before and after applying imaging weights and the best fit theory curves. The solid line and the grey shade represent respectively the mean power spectrum and 1$\sigma$ error, estimated from the $\fnl=0$ lognormal simulations. The differences between various cleaning methods are significant on large scales ($\ell > 20$), but the small scale clustering measurements are consistent. By comparing \textit{linear two maps} to \textit{linear three maps}, we find that the measured clustering power on modes with $6\leq \ell < 10$ are noticeably different between the two methods. We associate the differences to the additional map for psfsize in the r-band, which is included in \textit{linear three maps}. On other scales, the differences between \textit{linear three maps} and \textit{linear eight maps} are negligible, supporting the idea that our feature selection procedure has been effective in identifying the primary maps which cause the large-scale excess clustering signal. Comparing \textit{nonlinear three maps} to \textit{linear three maps}, we find that the measured spectra on $4 \leq \ell < 6$ are very different, probably indicating some nonlinear spurious fluctuations with large scale characteristics due to extinction. Including stellar density in the nonlinear approach (\textit{nonlinear four maps}) further reduces the excess power relative to the mock power spectrum, in particular on modes between $2\leq \ell < 4$. When calibrated on the lognormal simulations, we find that these differences are recovered after accounting for over-correction. Therefore, we associate this subtraction to over fitting.


\subsubsection{Calibrated constraints}

\begin{figure}
    \raggedleft
    \includegraphics[width=0.424\textwidth]{mcmc_dr9methods1d.pdf}
    \includegraphics[width=0.45\textwidth]{figures/mcmc_dr9methods.pdf} 
    \caption{The calibrated constrains from the DR9 LRG targets. \textit{Top}: probability distribution for $\fnl$ marginalized over the shotnoise and bias. \textit{Bottom}: $68\%$ and $95\%$ probability distribution contours for the bias and $\fnl$ from the DR9 LRG sample before and after applying nonlinear cleaning methods. The lognormal mocks are used to calibrate these distributions for over correction.}\label{fig:mcmc_dr9}
\end{figure}

\begin{table*}
    \caption{The calibrated best fit, marginalized mean, and marginalized $68\%$ ($95\%$) confidence estimates for $\fnl$ from fitting the DR9 LRG power spectrum before and after correcting for imaging systematic effects.}
    \label{tab:dr9methodcalib}
   \centerline{%     
    \begin{tabular}{llllllll}
    \hline
    \hline
   &  & 	  & & $\fnl$ &  &  \\
   \cmidrule(r{.7cm}){3-6}
Footprint                               & Method & 	Best fit  & Mean & $ 68\%$ CL & $ 95\%$ CL & $\chi^{2}$/dof \\
    \hline
DESI                      & No Weight   & $113.18$& $115.49$& $ 98.14<\fnl<132.89$& $ 83.51<\fnl<151.59$ &   44.4/34\\
DESI                      & Nonlinear Three Maps& $ 47.38$& $ 48.81$& $ 36.08<\fnl< 61.44$& $ 25.03<\fnl< 75.64$ &   34.6/34\\
DESI                      & Nonlinear Four Maps& $ 48.92$& $ 50.10$& $ 36.88<\fnl< 63.31$& $ 24.87<\fnl< 77.78$ &   35.2/34\\
DESI                      & Nonlinear Nine Maps& $ 49.69$& $ 41.91$& $ 13.10<\fnl< 69.14$& $-15.96<\fnl< 91.84$ &   39.5/34\\
   \hline
    \end{tabular}
}
\end{table*}

All $\fnl$ constraints presented here are calibrated for the effect of over correction using the lognormal simulations. Table \ref{tab:dr9methodcalib} describes the best fit and marginalized mean estimates of $\fnl$ from fitting the power spectrum of the DR9 LRG sample before and after cleaning with the nonlinear approach given various combinations for the imaging systematic maps. Figure \ref{fig:mcmc_dr9} shows the marginalized probability distribution for $\fnl$ in the top panel, and the $68\%$ and $95\%$ probability contours for the linear bias parameter and $\fnl$ in the bottom panel, from our sample before and after applying various corrections for imaging systematics. Overall, we find the maximum likelihood estimates to be consistent among the various cleaning methods. We obtain $36.08 (25.03) < \fnl < 61.44(75.64)$ with $\chi^{2}=34.6$ for \textit{nonlinear three maps} over 34 degrees of freedom. Accounted for over-correction, we obtain $36.88(24.87) < \fnl < 63.31(77.78)$ with $\chi^{2}=35.2$ with the additional stellar density map in the \textit{nonlinear four maps}. With or without $nStar$, the confidence intervals are consistent with each other and more than $3\sigma$ off from zero PNG. Cleaning the sample with \textit{nonlinear nine maps} weakens our constraints to $13.10(-15.96) < \fnl < 69.14(91.84)$ with $\chi^{2}=39.5$. For comparison, we obtain $98.14(83.51) < \fnl < 132.89(151.59)$ at $68\% (95\%)$ confidence with $\chi^{2}=44.4$ for the \textit{no weight} approach. The uncalibrated probability contours are presented in Appendix \ref{sec:dr9uncalib}.


\subsubsection{Uncalibrated constraints: robustness tests}
Now we proceed to perform some robustness tests and assess how sensitive the $\fnl$ constraints are to the assumptions made in the analysis or the quality cuts applied to the data. For each case, we re-train the cleaning methods and derive new sets of imaging weights. Accordingly, for the cases where a new survey mask is applied to the data, we re-calculate the covariance matrices using the new survey mask to account for the changes in the survey window and integral constraint effects. Calibrating the mitigation biases for all of these experiments is beyond the scope of this work and redundant, as we are only interested in the relative shift in the $\fnl$ constraints after changing the assumptions. Therefore, the absolute scaling of the $\fnl$ constraints presented here are biased because of the over correction effect. Table \ref{tab:dr9method} summarizes the uncalibrated $\fnl$ constraints from the DR9 LRG sample. Our tests are as follows:

\begin{table*}
    \caption{The uncalibrated best fit and marginalized mean estimates for $\fnl$ from fitting the power spectrum of the DR9 LRG sample before and after correcting for systematics. The number of degrees of freedom is 34 (37 data points - 3 parameters). The lowest mode is $\ell=2$ and the covariance matrix is from the $\fnl=0$ clean mocks (no mitigation) except for the case with '+cov' in which the covariance matrix is from the $\fnl=76.9$ clean mocks (no mitigation).}
    \label{tab:dr9method}
   \centerline{%     
    \begin{tabular}{llllllll}
    \hline
    \hline
   &  & 	  & & $\fnl$ &  &  \\
   \cmidrule(r{.7cm}){3-6}
Footprint                               & Method & 	Best fit  & Mean & $ 68\%$ CL & $ 95\%$ CL & $\chi^{2}$ \\
    \hline
DESI                      & No Weight   & $113.18$& $115.49$& $ 98.14<\fnl<132.89$& $ 83.51<\fnl<151.59$ &   44.4\\
DESI                      & Linear Eight Maps& $ 36.05$& $ 37.72$& $ 26.13<\fnl< 49.21$& $ 16.31<\fnl< 62.31$ &   41.1\\
DESI                      & Linear Two Maps& $ 49.58$& $ 51.30$& $ 38.21<\fnl< 64.33$& $ 27.41<\fnl< 78.91$ &   38.8\\
DESI                      & Linear Three Maps& $ 36.63$& $ 38.11$& $ 26.32<\fnl< 49.86$& $ 16.36<\fnl< 63.12$ &   39.6\\
DESI                      & Nonlinear Three Maps& $ 28.58$& $ 29.79$& $ 18.91<\fnl< 40.59$& $  9.47<\fnl< 52.73$ &   34.6\\
DESI (imag. cut)          & Nonlinear Three Maps& $ 29.16$& $ 30.57$& $ 19.05<\fnl< 42.18$& $  9.01<\fnl< 54.81$ &   35.8\\
DESI (comp. cut)          & Nonlinear Three Maps& $ 28.07$& $ 29.48$& $ 18.38<\fnl< 40.50$& $  8.81<\fnl< 53.10$ &   34.5\\
DESI                      & Nonlinear Four Maps& $ 16.63$& $ 17.52$& $  7.51<\fnl< 27.53$& $ -1.59<\fnl< 38.49$ &   35.2\\
DESI                      & Nonlinear Nine Maps& $ -5.87$& $ -9.19$& $-21.45<\fnl<  2.40$& $-33.81<\fnl< 12.06$ &   39.5\\
DESI                     & Nonlinear Three Maps+$f_{\rm NL}=76.92$ Cov& $ 31.62$& $ 33.11$& $ 20.94<\fnl< 45.24$& $ 10.56<\fnl< 59.16$ &   33.5\\
\hline
BASS+MzLS                 & Nonlinear Three Maps& $ 15.43$& $ 19.01$& $ -1.17<\fnl< 39.43$& $-19.19<\fnl< 63.56$ &   35.6\\
BASS+MzLS                 & Nonlinear Four Maps& $ 13.12$& $ 15.39$& $ -4.59<\fnl< 35.56$& $-24.88<\fnl< 59.31$ &   34.7\\
BASS+MzLS                 & Nonlinear Nine Maps& $ -3.73$& $ -6.34$& $-27.11<\fnl< 13.75$& $-47.44<\fnl< 33.94$ &   36.8\\
BASS+MzLS (imag. cut)     & Nonlinear Three Maps& $ 25.03$& $ 29.12$& $  6.16<\fnl< 52.44$& $-14.22<\fnl< 80.54$ &   36.2\\
BASS+MzLS (comp. cut)     & Nonlinear Three Maps& $ 16.99$& $ 20.90$& $  0.26<\fnl< 41.76$& $-18.30<\fnl< 67.12$ &   35.8\\
DECaLS North              & Nonlinear Three Maps& $ 41.02$& $ 44.89$& $ 23.33<\fnl< 66.78$& $  4.96<\fnl< 93.02$ &   41.1\\
DECaLS North              & Nonlinear Four Maps& $ 31.45$& $ 34.78$& $ 14.14<\fnl< 55.79$& $ -5.81<\fnl< 80.80$ &   41.2\\
DECaLS North              & Nonlinear Five Maps& $ 55.46$& $ 60.44$& $ 36.78<\fnl< 84.05$& $ 17.86<\fnl<112.81$ &   38.4\\
DECaLS North              & Nonlinear Nine Maps& $  0.81$& $ -5.68$& $-29.73<\fnl< 16.71$& $-53.15<\fnl< 36.19$ &   45.1\\
DECaLS North (no DEC cut) & Nonlinear Three Maps& $ 41.05$& $ 44.82$& $ 23.58<\fnl< 66.08$& $  6.40<\fnl< 91.42$ &   40.7\\
DECaLS North (imag. cut)  & Nonlinear Three Maps& $ 43.27$& $ 48.39$& $ 24.60<\fnl< 72.50$& $  4.71<\fnl<101.42$ &   35.1\\
DECaLS North (comp. cut)  & Nonlinear Three Maps& $ 40.55$& $ 44.63$& $ 22.41<\fnl< 67.11$& $  3.95<\fnl< 94.06$ &   41.4\\
DECaLS South              & Nonlinear Three Maps& $ 31.24$& $ 33.21$& $ 14.89<\fnl< 52.40$& $ -5.11<\fnl< 74.35$ &   30.2\\
DECaLS South              & Nonlinear Four Maps& $ 14.34$& $  6.28$& $-21.19<\fnl< 30.01$& $-53.63<\fnl< 49.51$ &   31.9\\
DECaLS South              & Nonlinear Five Maps& $ 33.79$& $ 37.50$& $ 17.71<\fnl< 57.42$& $ -0.31<\fnl< 80.94$ &   30.8\\
DECaLS South              & Nonlinear Nine Maps& $-36.76$& $-32.01$& $-49.38<\fnl<-13.61$& $-65.26<\fnl<  7.52$ &   31.5\\
DECaLS South (no DEC cut) & Nonlinear Three Maps& $ 43.79$& $ 46.79$& $ 30.16<\fnl< 63.41$& $ 16.38<\fnl< 82.72$ &   23.8\\
DECaLS South (imag. cut)  & Nonlinear Three Maps& $ 26.47$& $ 23.36$& $  3.18<\fnl< 47.84$& $-57.69<\fnl< 71.39$ &   30.0\\
DECaLS South (comp. cut)  & Nonlinear Three Maps& $ 29.62$& $ 31.76$& $ 13.00<\fnl< 51.58$& $ -9.78<\fnl< 74.28$ &   29.7\\
   \hline
    \end{tabular}}
\end{table*}

\begin{figure}
    \centering
    \includegraphics[width=0.45\textwidth]{figures/mcmc_dr9regions.pdf} 
    \caption{The uncalibrated 2D constraints from the DR9 LRG sample for each imaging survey compared with that for the whole DESI footprint. The dark and light shades represent the $68\%$ and $95\%$ confidence intervals, respectively.}\label{fig:mcmc_dr9reg}
\end{figure}
\begin{figure*}
    \centering
    \includegraphics[width=0.85\textwidth]{figures/cldr9_lowell.pdf}
    \includegraphics[width=0.86\textwidth]{figures/fnl_elmin.pdf}  
    \caption{Top: The measured power spectra before and after \textit{nonlinear three maps}. The uncalibrated $\fnl$ constraints vs the lowest $\ell$ mode from the DR9 LRG sample cleaned with \textit{nonlinear three maps}. The points represent marginalized mean estimates of $\fnl$ and error bars represent $68$\% confidence.}\label{fig:mcmc_dr9elmin}
\end{figure*}

\begin{itemize}[itemindent=*]

\item \textbf{Linear methods}: We find consistent constraints from \textit{linear eight maps} and \textit{linear three maps} which suggests that not all imaging systematic maps are needed to completely mitigate systematic effects. We find $\sigma (\fnl) \sim 25$ for the linear methods. With the same set of imaging systematic maps, the nonlinear method yields a smaller constraint, $\Delta \fnl = -8$, and a better $\chi^{2}$ fit, e.g., $34.6$ vs $39.6$ for 34 degrees of freedom, even though the covariance matrix is fixed.

\item \textbf{Imaging regions}: We compare how our constraints from fitting the power spectrum of the whole DESI footprint compares to that from the power spectrum of each imaging region individually, namely BASS+MzLS, DECaLS North, and DECaLS South. Figure~\ref{fig:mcmc_dr9reg} shows the $68\%$ and $95\%$ probability contours on $\fnl$ and $b$ from each individual region, compared with that from DESI. The cleaning method here is \textit{nonlinear three maps}, and the covariance matrices are estimated from the $\fnl=0$ mocks. Overall, we find that the constraints from all imaging surveys are consistent with each other and DESI within $68\%$ confidence. Both BASS+MzLS and DECaLS South yield constraints consistent with $\fnl=0$ within $95\%$, but DECaLS North deviates from zero PNG at more than $2\sigma$. This motivates follow-up studies with the spectroscopic sample of LRGs in DECaLS North.

\item \textbf{Stellar density template (\textit{nStar})}: Adding the stellar density template (\textit{nonlinear four maps}) does not change the constraints from BASS+MzLS much, but it shifts the $\fnl$ distributions to lower values in DECaLS North and DECaLS South by $0.5\sigma$ and $\sigma$, respectively, reconciling all constraints with $\fnl=0$. We note that differences are more significant when all nine maps are used as input. This is somewhat expected as cleaning the data with more imaging systematic maps is more prone to the over-correction issue. \mr{We find that the shifts in $\fnl$ from adding $nStar$ are cancelled after accounting for the over correction bias. Comparing \textit{nonlinear four maps} and \textit{nonlinear three maps} in Table \ref{tab:dr9methodcalib} and \ref{tab:dr9method} is indicative that the $\fnl$ shifts after adding $nStar$ are indicative of over-correction due to correlations between the stellar density template and large-scale structure.} 

\item \textbf{Pixel completeness (\textit{comp. cut})}: We discard pixels with fractional completeness less than half to assess the effect of partially complete pixels on $\fnl$. This cut removes $0.6\%$ of the survey area, and no changes in the $\fnl$ constraints are observed.

\item \textbf{Imaging quality (\textit{imag. cut})}: Pixels with poor photometry are removed from our sample by applying the following cuts on imaging; $E[B-V]<0.1$, $nStar < 3000$, ${\rm depth}_{g} > 23.2$, ${\rm depth}_{r} > 22.6$, ${\rm depth}_{z} > 22.5$, ${\rm psfsize}_{g}<2.5$, ${\rm psfsize}_{r}<2.5$, and ${\rm psfsize}_{z}<2$. Although these cuts remove $8\%$ of the survey mask, there is a negligible impact on the best fit $\fnl$ from fitting the DESI power spectrum. However, when each region is fit individually, the BASS+MzLS constraint shift toward higher values of $\fnl$ by approximately $\Delta \fnl \sim 10$, whereas the constraints from DECaLS North and DECaLS South do not change significantly. 

\item \textbf{Covariance matrix (\textit{cov})}: We fit the power spectrum of our sample cleaned with \textit{nonlinear three maps} correction, but use the covariance matrix constructed from the $\fnl=76.92$ mocks. With the alternative covariance, a $12\%$ increase in the $\sigma \fnl$ is observed. We also find that the best fit and marginalized mean estimates of $\fnl$ increase by $10-11\%$. Overall, we find that the differences are not significant in comparison to the statistical precision.

\item \textbf{External maps (\textit{CALIBZ+HI})}: The neural network five maps correction includes the additional maps for HI and CALIBZ. With this correction, the best fit $\fnl$ increases from $41.02$ to $55.46$ for DECaLS North and from $31.24$ to $33.79$ for DECaLS South, which might suggest that adding HI and CALIBZ increases the input noise, and thus negatively impacts the performance of the neural network model. This test is not performed on BASS+MzLS due to a lack of coverage from the CALIBZ map. 

\item \textbf{Declination mask (\textit{no DEC cut})}: The fiducial mask removes the disconnected islands in DECaLS North and regions with DEC $<-30$ in DECaLS South, where there is a high likelihood of calibration issues as different standard stars are used for photometric calibrations. We analyze our sample without these cuts, and find that the best fit and marginalized $\fnl$ mean estimates from DECaLS South shift significantly to higher values of $\fnl$ by $\Delta \fnl \sim 10$, which supports the issue of photometric systematics in the DECaLS South region below DEC $=-30$. On the other hand, the constraints from DECaLS North do not change significantly, indicating the islands do not induce significant contaminations. \bbk{[Did you try cutting out more of DECaLS North, such as everything below the equator?]}

\item \textbf{Scale dependence (\textit{varying $\ell_{\rm min}$})}: We raise the value of the lowest harmonic mode $\ell_{\rm min}$ used for the likelihood evaluation during MCMC. This is equivalent to decreasing the highest scale of measurement in the power spectrum. By doing so, we anticipate a reduction in the impact of imaging systematics on $\fnl$ inference as lower $\ell$ modes are more likely to be contaminated. Figure \ref{fig:mcmc_dr9elmin} illustrates the power spectra before and after the correction with \textit{nonlinear three maps} in the top panel. The bottom panel shows the marginalized mean and $68\%$ error on $\fnl$ with \textit{nonlinear three maps} for the DESI, BASS+MzLS, DECaLS North, and DECaLS South regions. \mr{ALEX: We find that the mean estimates of $\fnl$ slightly shifts to higher values on scales $12<\ell<18$ in DECaLS North and BASS+MzLS when higher $\ell_{\rm min}$ is used. This is the opposite behavior from what one would expect if there were just a giant systematics induced spike at low $\ell$. So it shows that the issue here is more subtle than what one would have initially suspected.}

\end{itemize}

\subsection{Summary}
In summary, we find that the nonlinear methods outperform the linear methods in removing the excess clustering signal on large scales. Adding the stellar density map results in significant changes, however when accounted for the mitigation bias, all methods recover the same maximum likelihood estimate. With calibration on the lognormal mocks, the conservative approaches that use a small subset of imaging systematic maps show $\fnl$ detection at more than $2\sigma$ confidence. The most flexible nonlinear method with nine maps returns a bigger associated uncertainty which is consistent with $\fnl=0$. We also run various tests with cuts on the DR9 sample or changing the configuration or details of the analysis. Overall, we find consistent results across sub imaging surveys within DESI. However, our results show that a declination cut at DEC $=-30$ is necessary for DECaLS South to avoid potential calibration issues. Our analysis does not show a statistical demand for including external templates for HI and CALIBZ, using a different covariance matrix, or imposing additional cuts on the DR9 based on imaging and pixel completeness. We also obtain robust results regardless of the largest scale used for constraining $\fnl$.
\section{Discussion and Conclusion}\label{sec:conclusion}

We have measured the local PNG parameter $\fnl$ using the scale-dependent bias in the angular clustering of LRGs selected from the DESI Legacy Imaging Survey DR9. Our sample includes more than $12$ million LRG targets covering around $14,000$ square degrees in the redshift range of $0.2< z < 1.35$. We leverage early spectroscopy during DESI Survey Validation \citep{desi2023sv} to infer the redshift distribution of our sample (Figure \ref{fig:nz}). \mr{Our power spectrum model accounts for various theoretical and observational effects such as RSD, magnification bias, survey geometry, and integral constraint. Most importantly, we utilize novel a machine learning-method to mitigate the effect of imaging systematics and reduce excess clustering power at low $\ell$.} In our fiducial analysis, we have obtained a maximum likelihood value of \mr{$\fnl=46$} with a significant probability that $\fnl$ is greater than zero, $P(\fnl>0)=99.9$ (Figure \ref{fig:mcmc_dr9} and Table \ref{tab:dr9methodcalib}). \mr{Our findings are robust against parameters $p$ and $s$. Even without accounting for over-correction, we find significant deviation from zero PNG at $P(\fnl>0)=99.4$.  While we have not measured the mitigation over-correction for the linear cases, it would only make the result more significantly non-zero and we cannot envision this changing any of our conclusions.}

The signature of local PNG is very sensitive to excess clustering signals caused by imaging systematic effects. We have applied a series of robustness tests to investigate the impact of how the galaxy selection function is determined. Specifically, both linear and nonlinear methods are applied using various combinations of imaging systematic maps (Galactic extinction, stellar density, depth in $grzW1$, psfsize in $grz$, and neutral hydrogen column density). We also examine the effect of different masks based on imaging. Overall, we find no change in the analysis that shifts the maximum likelihood value of $\fnl$ to a significantly lower value (Figure \ref{fig:mcmc_dr9reg}, Figure \ref{fig:mcmc_dr9elmin}, and Table \ref{tab:dr9method}). The only manner in which the significance of nonzero PNG decreases is due to the uncertainty on the measurement increasing when we employ more imaging systematic maps for the selection function estimation and by doing so remove large-scale clustering information (the effect of which on $\fnl$ recovery we have calibrated with mocks, as shown in Figure \ref{fig:fnlbias}).

\begin{figure}
    \centering
    \includegraphics[width=0.45\textwidth]{figures/fnl_history.pdf}
    \caption{History of constraints on local PNG $\fnl$ at $95\%$ confidence from single-tracer LSS \citep{slosar2008constraints,2013MNRAS.428.1116R, mueller2022primordial, 2022PhRvD.106d3506C}, including our analysis with $\mr{21}<\fnl<76$ (DESI photo LRG) and CMB surveys \citep{Komatsu_2003, Komatsu_2010, planck13, akrami2019planck}. The median $\fnl$ value is used in case the maximum likelihood estimate was not reported in the reference.}
    \label{fig:fnlhist}
\end{figure}

When comparing our fiducial results to recent CMB and QSO measurements, as shown in Figure \ref{fig:fnlhist}, we find a significant tension with CMB \mr{at more than $3 \sigma$} but consistent constraints with LSS within $95\%$ confidence. \mr{We emphasize that the tension between our non-linear three maps constraint and CMB is persistent even without accounting for over-correction; $P(\fnl>0)$ decreases from 99.9 to 99.4 per cent or more than $2\sigma$. The uncalibrated constraints from the non-linear four and non-linear nine maps methods are consistent with CMB, but these methods are highly prone to over-fitting and their applications to the LRG sample are not justified based on our remaining systematic $\chi^{2}$ diagnostics.} Therefore, either we have measured an $\fnl$ signal that is inconsistent with CMB measurements or there is a hidden source of systematic contamination in our data which cannot be mitigated with available imaging systematic maps. \mr{There is a body of exciting work in the literature exploring models where $\fnl$ (or another non-Gaussianity parameter) varies with scales and evades the CMB constraint while showing up in the LSS \citep{2008JCAP...04..014L,sefusatti_2009, becker_2011, Becker_2012}. On the other hand, a major limitation of template-based mitigation techniques, like the methods employed in this study, is that the data cannot be corrected for unknown systematic effects, which we do not have a template for. In this regard, multi-tracer techniques can be considered as robust alternatives since they are less sensitive to systematics. Cross-correlations of DESI LRG targets with CMB lensing could shed more light on the issues discussed in this paper. Additionally, forward modeling techniques for estimating the galaxy survey selection function can provide a new path to handle unknown systematics \citep{suchyta2016, kong2020}.}

Our analysis can be considered as the first attempt to identify major systematics in DESI, so we can be ready for constraining $\fnl$ with DESI spectroscopy. Internal DESI tests of the photometric calibration were unable to uncover DESI-specific issues, e.g., when comparing to Gaia data. The most significant trends that we find are with the E(B-V) map. The source of such a trend would be a mis-calibration of the E(B-V) map itself or the coefficients applied to obtain Galactic extinction corrected photometry. Such a mis-calibration would plausibly be proportional in amplitude to the estimated E(B-V) map, though it may not have E(B-V)’s spatial distribution. In order to explain the $\fnl$ signal we measure, such an effect would need to be approximately twice that of the trend we find with E(B-V). There are ongoing efforts within DESI to obtain improved Galactic extinction information, which will help establish if this is indeed the cause.
\section*{Acknowledgements}
%We thank the anonymous referee for their insightful comments and suggestions. M.R. is supported by the U.S.~Department of Energy, Office of Science, Office of High Energy Physics under DE-SC0014329; H.-J.S. is supported by the U.S.~Department of Energy, Office of Science, Office of High Energy Physics under DE-SC0014329 and DE-SC0019091. We acknowledge the support and resources from the Ohio Supercomputer Center \citep[OSC;][]{Owens2016}. Specifically, this work utilized more than $359000$ core hours of the Owens cluster. M.R. is grateful for help from Xia Wang, Antonio Marcum, and Yu Feng. G.R. acknowledges support from the National Research Foundation of Korea (NRF) through Grants No. 2017R1E1A1A01077508 and No. 2020R1A2C1005655 funded by the Korean Ministry of Education, Science and Technology (MoEST). We would like to appreciate the open-source software and modules that were invaluable to this research: Pytorch, Nbodykit, HEALPix, Fitsio, Scikit-Learn, NumPy, SciPy, Pandas, IPython, Jupyter, and GitHub.

%Funding for the Sloan Digital Sky 
%Survey IV has been provided by the 
%Alfred P. Sloan Foundation, the U.S. 
%Department of Energy Office of 
%Science, and the Participating 
%Institutions. SDSS-IV acknowledges support and 
%resources from the Center for High 
%Performance Computing  at the 
%University of Utah. The SDSS 
%website is www.sdss.org. This work also relied on resources provided to the eBOSS Collaboration by the National Energy Research Scientific Computing Center (NERSC). NERSC is a U.S. Department of Energy Office of Science User Facility operated under Contract No. DE-AC02-05CH11231.
% 
%
%SDSS-IV is managed by the 
%Astrophysical Research Consortium 
%for the Participating Institutions 
%of the SDSS Collaboration including 
%the Brazilian Participation Group, 
%the Carnegie Institution for Science, 
%Carnegie Mellon University, Center for 
%Astrophysics | Harvard \& 
%Smithsonian, the Chilean Participation 
%Group, the French Participation Group, 
%Instituto de Astrof\'isica de 
%Canarias, The Johns Hopkins 
%University, Kavli Institute for the 
%Physics and Mathematics of the 
%Universe (IPMU) / University of 
%Tokyo, the Korean Participation Group, 
%Lawrence Berkeley National Laboratory, 
%Leibniz Institut f\"ur Astrophysik 
%Potsdam (AIP),  Max-Planck-Institut 
%f\"ur Astronomie (MPIA Heidelberg), 
%Max-Planck-Institut f\"ur 
%Astrophysik (MPA Garching), 
%Max-Planck-Institut f\"ur 
%Extraterrestrische Physik (MPE), 
%National Astronomical Observatories of 
%China, New Mexico State University, 
%New York University, University of 
%Notre Dame, Observat\'ario 
%Nacional / MCTI, The Ohio State 
%University, Pennsylvania State 
%University, Shanghai 
%Astronomical Observatory, United 
%Kingdom Participation Group, 
%Universidad Nacional Aut\'onoma 
%de M\'exico, University of Arizona, 
%University of Colorado Boulder, 
%University of Oxford, University of 
%Portsmouth, University of Utah, 
%University of Virginia, University 
%of Washington, University of 
%Wisconsin, Vanderbilt University, 
%and Yale University.
 
\section*{Data Availability}
\label{sec:dataavail}

The catalogue described in this article will be made public in Github at \url{https://github.com/mehdirezaie/eBOSSDR16QSOE}. The neural network pipeline utilized in this work is publicly available at \url{https://github.com/mehdirezaie/sysnetdev}.

% references
\bibliographystyle{mnras}
\bibliography{refs} 
\appendix
\section{Redshift distribution}
Redshift distribution of LRGs is constructed from the DESI SV data release of Denali with the same selection. The fiducial distribution only covers the redshift range from 0.2 to 1.35. Below we test the impact of LRG dN/dz on the angular power spectrum.


\begin{figure}
\centering
\includegraphics[width=0.45\textwidth]{nztreat.pdf}
\includegraphics[width=0.45\textwidth]{cell_nz.pdf}
\caption{Top: Redshift distribution of LRGs. Bottom: Power spectrum given various dN/dz treatments for two arbitrary $\fnl$ values.}
\end{figure}


\section{Scale dependence systematics}
The default modes used in calculating the cross spectrum $\chi^{2}$ diagnostic range for $2 \leq \ell < 20$. Here we further test the stability of our results by extending the highest mode out to $\ell=100$ or fluctuations over scales as small as $1.8$ degrees. 
\begin{figure}
\centering
\includegraphics[width=0.45\textwidth]{chi2lmax.pdf}
\caption{Cross Spectrum $\chi^{2}$ as a function of the highest mode $\ell_{\rm max}$. The lowest mode is $\ell_{\rm min}=2$.}
\end{figure}

% % % Don't change these lines
\bsp	% typesetting comment
\label{lastpage}

\end{document}
% End of mnras_template.tex