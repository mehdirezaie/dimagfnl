\section*{Acknowledgements}
MR is supported by the U.S. Department of Energy grants DE-SC0021165 and DE-SC0011840. H-JS acknowledges support from the U.S. Department of Energy, Office of Science, 
Office of High Energy Physics under grant No. DE-SC0019091 and No. DE-SC0023241. FB is a University Research Fellow, and has received funding from the European Research Council (ERC) under the European Union’s Horizon 2020 research and innovation program (grant agreement 853291). BB-K is supported by the project \begin{CJK}{UTF8}{mj}우주거대구조를 이용한 암흑우주 연구\end{CJK} (``Understanding Dark Universe Using Large Scale Structure of the Universe’’), funded by the Ministry of Science of the Republic of Korea. We acknowledge the support and resources from the Ohio Supercomputer Center \citep[OSC;][]{OhioSupercomputerCenter1987}.  


MR would like to thank CCAPP, in particular, John Beacom and Lisa Colarosa, for the hospitality and support. We would like to thank Tanveer Karim and Sukhdeep Singh for helpful discussions, and Rongpu Zhou for providing the galaxy density and imaging systematic maps. This research has made substantial use of the arXiv preprint server, NASA’s Astrophysics Data System, Github's online software development platform, and many open-source software, such as Pytorch, Nbodykit, HEALPix, Fitsio, Scikit-Learn, NumPy, SciPy, Pandas, IPython, and Jupyter. 


The DESI research is supported by the Director, Office of Science, Office of High Energy Physics of the U.S. Department of Energy under Contract No. DE–AC02–05CH11231, and by the National Energy Research Scientific Computing Center, a DOE Office of Science User Facility under the same contract; additional support for DESI is provided by the U.S. National Science Foundation, Division of Astronomical Sciences under Contract No. AST-0950945 to the NSF’s National Optical-Infrared Astronomy Research Laboratory; the Science and Technologies Facilities Council of the United Kingdom; the Gordon and Betty Moore Foundation; the Heising-Simons Foundation; the French Alternative Energies and Atomic Energy Commission (CEA); the National Council of Science and Technology of Mexico (CONACYT); the Ministry of Science and Innovation of Spain (MICINN), and by the DESI Member Institutions: \href{https://www.desi.lbl.gov/collaborating-institutions}{www.desi.lbl.gov/collaborating-institutions}. 

The DESI Legacy Imaging Surveys consist of three individual and complementary projects: the Dark Energy Camera Legacy Survey (DECaLS), the Beijing-Arizona Sky Survey (BASS), and the Mayall z-band Legacy Survey (MzLS). DECaLS, BASS and MzLS together include data obtained, respectively, at the Blanco telescope, Cerro Tololo Inter-American Observatory, NSF’s NOIRLab; the Bok telescope, Steward Observatory, University of Arizona; and the Mayall telescope, Kitt Peak National Observatory, NOIRLab. NOIRLab is operated by the Association of Universities for Research in Astronomy (AURA) under a cooperative agreement with the National Science Foundation. Pipeline processing and analyses of the data were supported by NOIRLab and the Lawrence Berkeley National Laboratory. Legacy Surveys also uses data products from the Near-Earth Object Wide-field Infrared Survey Explorer (NEOWISE), a project of the Jet Propulsion Laboratory/California Institute of Technology, funded by the National Aeronautics and Space Administration. Legacy Surveys was supported by: the Director, Office of Science, Office of High Energy Physics of the U.S. Department of Energy; the National Energy Research Scientific Computing Center, a DOE Office of Science User Facility; the U.S. National Science Foundation, Division of Astronomical Sciences; the National Astronomical Observatories of China, the Chinese Academy of Sciences and the Chinese National Natural Science Foundation. LBNL is managed by the Regents of the University of California under contract to the U.S. Department of Energy. The complete acknowledgments can be found at \href{https://www.legacysurvey.org/}{www.legacysurvey.org/}. The authors are honored to be permitted to conduct scientific research on Iolkam Du’ag (Kitt Peak), a mountain with particular significance to the Tohono O’odham Nation.