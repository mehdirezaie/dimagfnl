\section{Results}\label{sec:results}
\subsection{DESI imaging LRG sample}

Fig. \ref{fig:cl_dr9} shows the measured power spectrum of the DESI imaging DR9 LRG sample before and after applying imaging weights, the best fit theory curves, and the mean power spectrum and 1$\sigma$ error estimated from the $\fnl=0$ lognormal simulations. The power spectra are similar on small scales ($\ell > 20$), but the differences between various cleaning methods are significant on large scales. By comparing \textit{linear conservative I} to \textit{linear conservative II}, we find that the measure power spectra on modes with $6\leq \ell < 10$ are noticeably different between the two methods. We associate the differences to the r-band psfsize template in \textit{linear conservative II}. On other scales, the differences between the spectra after the linear-based cleaning are negligible, supporting the idea that our feature selection procedure has worked to identify the primary maps causing excess clustering signal. Comparing \textit{nonlinear conservative II} to \textit{linear conservative II}, we find that the measured spectra on $4 \leq \ell < 6$ are very different, probably pointing at nonlinear spurious fluctuations with large-scale characteristics due to the extinction. Adding stellar density to the nonlinear approach (\textit{nonlinear conservative II + nStar}) results in less excess power relative to the mock power spectrym, with the modes on $2\leq \ell < 4$ reflecting the biggest change. The flexibility of the nonlinear approach to correct for these effects ameliorates the clustering measurements on these scales.

\begin{figure}
    \centering
    \includegraphics[width=0.48\textwidth]{figures/model_dr9.pdf} 
    \caption{The angular power spectrum of the DR9 LRG sample before (\textit{No weight}) and after correcting for imaging systematics using various methods with their corresponding best fit theory curves. The shade represents $1\sigma$ error constructed from the $\fnl=0$ mocks.}
    \label{fig:cl_dr9}
\end{figure}

\subsubsection{Calibrated constraints}

\begin{figure}
    \raggedleft
    \includegraphics[width=0.424\textwidth]{mcmc_dr9methods1d.pdf}
    \includegraphics[width=0.45\textwidth]{figures/mcmc_dr9methods.pdf} 
    \caption{Calibrated constrains from the DR9 LRG sample. \textit{Top}: probability distribution for $\fnl$ marginalized over the shotnoise and bias. \textit{Bottom}: $68\%$ and $95\%$ probability distribution contours for the bias and $\fnl$ from the DR9 LRG sample before and after applying nonlinear cleaning methods. The lognormal mocks are used to correct these distributions for mitigation bias.}\label{fig:mcmc_dr9}
\end{figure}

\begin{table*}
    \caption{Calibrated best fit and marginalized mean estimates for $\fnl$ from fitting power spectrum of the DESI DR9 LRG sample before and after correcting for systematics. Degree of freedom is 34 (37 data points - 3 parameters).}
    \label{tab:dr9methodcalib}
   \centerline{%     
    \begin{tabular}{llllllll}
    \hline
    \hline
   &  & 	  & & $\fnl$ &  &  \\
   \cmidrule(r{.7cm}){3-6}
Footprint                               & Method & 	Best fit  & Mean & $ 68\%$ CL & $ 95\%$ CL & $\chi^{2}$ \\
    \hline
DESI                      & No Weight   & $113.18$& $115.49$& $ 98.14<\fnl<132.89$& $ 83.51<\fnl<151.59$ &   44.4\\
DESI                      & Nonlinear (Cons. II)& $ 47.38$& $ 48.81$& $ 36.08<\fnl< 61.44$& $ 25.03<\fnl< 75.64$ &   34.6\\
DESI                      & Nonlin. (Cons. II+nStar)& $ 48.92$& $ 50.10$& $ 36.88<\fnl< 63.31$& $ 24.87<\fnl< 77.78$ &   35.2\\
DESI                      & Nonlin. (All Maps+nStar)& $ 49.69$& $ 41.91$& $ 13.10<\fnl< 69.14$& $-15.96<\fnl< 91.84$ &   39.5\\
   \hline
    \end{tabular}
}
\end{table*}


Tab. \ref{tab:dr9methodcalib} describes the best fit and marginalized mean estimates of $\fnl$ from fitting the power spectrum of the DR9 LRG sample before and after cleaning with the nonlinear approach given various combinations of imaging templates. All constraints are calibrated for the effect of mitigation bias using the lognormal simulations. With the corrections applied, we obtain $36.08 (25.03) < \fnl < 61.44(75.64)$ with $\chi^{2}=34.6$ for \textit{nonlinear conservative II}, $36.88(24.87) < \fnl < 63.31(77.78)$ with $\chi^{2}=35.2$ for \textit{nonlinear conservative II + nStar}, and $13.10(-15.96) < \fnl < 69.14(91.84)$ with $\chi^{2}=39.5$ for \textit{nonlinear all maps + nStar} at $68\% (95\%)$ confidence over 34 degrees of freedom. \mr{here}

Fig \ref{fig:mcmc_dr9} shows the marginalized probability distribution for $\fnl$ (top) and the $68\%$ and $95\%$ probability contours for the bias parameter and $\fnl$ from the DR9 sample before (\textit{no weight}) and after applying different cleaning schemes. No weight constraint at $68\%$ is $98.14<\fnl<132.89$ with a best fit of $113.18$ and marginalized mean of $115.49$, and is more than $2\sigma$ off from zero. Applying imaging weights shifts constraints to lower $\fnl$ values, and $b$ is slightly pulled upward since excess clustering due to systematics is removed. Using all maps with the linear model does not change the results, showing that three maps are sufficient at the linear level to mitigate systematics.  As an alternative, using a nonlinear model with three maps shows around $1\sigma$ shift, with $68\%$ confidence at $18.91<\fnl<40.59$, inconsistent with zero for more than $2\sigma$. Adding a template for the local stellar density shift constraints by $1\sigma$, making it consistent with zero. As the most rigorous approach, using all maps and stellar density included results in more than $2\sigma$ shift. We emphasize that these shifts to lower $\fnl$ are somewhat expected as more input maps results in regressing more modes from cosmological clustering signal. Therefore, we use lognormal mocks to calibrate the amount of signal that is removed in each case and attempt to undo the effect.



\subsubsection{Robustness tests}

These results are subject to mitigation bias. $\fnl$ constraints from DR9 LRG sample is summarized in Table \ref{tab:dr9method}. First, we focus on the DESI footprint and then compare constraints obtained from each sub-survey. We also evaluate the robustness of constraints against various cuts and configurations. First, we compare how constraints from whole DESI footprint compares to those from each survey individually, namely BASS+MzLS, DECaLS Nouth, and DECaLS South. Fig. \ref{fig:mcmc_dr9reg} shows $68\%$ and $95\%$ confidence on $\fnl$ and $b$ from each individual survey or all combined as DESI. Constraints from all surveys are consistent and agree with each other within $68\%$. Both BASS+MzLS and DECaLS South are consistent with zero PNG, but DECaLS North deviates from zero at more than $2\sigma$. Adding the stellar density template does not change constraints from BASS+MzLS much, but it shifts DECaLS North and DECaLS South by $0.5\sigma$ and $\sigma$, respectively. This might indicate that there are some unresolved issues with stellar contamination in DECaLS North and DECaLS South. We note that differences are more significant when all maps and stellar density are used as input. This is expected as more maps mean the model has more freedom to take out clustering modes.


\begin{table*}
    \caption{Uncalibrated best fit and marginalized mean estimates for $\fnl$ from fitting power spectrum of the DR9 LRG sample before and after correcting for systematics. Degree of freedom is 34 (37 data points - 3 parameters).}
    \label{tab:dr9method}
   \centerline{%     
    \begin{tabular}{llllllll}
    \hline
    \hline
   &  & 	  & & $\fnl$ &  &  \\
   \cmidrule(r{.7cm}){3-6}
Footprint                               & Method & 	Best fit  & Mean & $ 68\%$ CL & $ 95\%$ CL & $\chi^{2}$ \\
    \hline
DESI                                    & No Weight   & $113.18$& $115.49$& $ 98.14<\fnl<132.89$& $ 83.51<\fnl<151.59$ &   44.4\\
DESI                                    & Linear (All Maps)& $ 36.05$& $ 37.72$& $ 26.13<\fnl< 49.21$& $ 16.31<\fnl< 62.31$ &   41.1\\
DESI                                    & Linear (Conservative I)& $ 49.58$& $ 51.30$& $ 38.21<\fnl< 64.33$& $ 27.41<\fnl< 78.91$ &   38.8\\
DESI                                    & Linear (Conservative II)& $ 36.63$& $ 38.11$& $ 26.32<\fnl< 49.86$& $ 16.36<\fnl< 63.12$ &   39.6\\
DESI                                    & Nonlinear (Cons. II)& $ 28.58$& $ 29.79$& $ 18.91<\fnl< 40.59$& $  9.47<\fnl< 52.73$ &   34.6\\
DESI                                    & Nonlin. (Cons. II+nStar)& $ 16.63$& $ 17.52$& $  7.51<\fnl< 27.53$& $ -1.59<\fnl< 38.49$ &   35.2\\
DESI                                    & Nonlin. (All Maps+nStar)& $ -5.87$& $ -9.19$& $-21.45<\fnl<  2.40$& $-33.81<\fnl< 12.06$ &   39.5\\
DESI (imag. cut)                  & Nonlin. (Cons. II)& $ 29.16$& $ 30.57$& $ 19.05<\fnl< 42.18$& $  9.01<\fnl< 54.81$ &   35.8\\
DESI (comp. cut)                 & Nonlin. (Cons. II)& $ 28.07$& $ 29.48$& $ 18.38<\fnl< 40.50$& $  8.81<\fnl< 53.10$ &   34.5\\
DESI                                    & Nonlin. (Cons. II)+Cov& $ 31.62$& $ 33.11$& $ 20.94<\fnl< 45.24$& $ 10.56<\fnl< 59.16$ &   33.5\\
\hline
BASS+MzLS                        & Nonlin. (Cons. II)& $ 15.43$& $ 19.01$& $ -1.17<\fnl< 39.43$& $-19.19<\fnl< 63.56$ &   35.6\\
BASS+MzLS                        & Nonlin. (Cons. II+nStar)& $ 13.12$& $ 15.39$& $ -4.59<\fnl< 35.56$& $-24.88<\fnl< 59.31$ &   34.7\\
BASS+MzLS                        & Nonlin. (All Maps+nStar)& $ -3.73$& $ -6.34$& $-27.11<\fnl< 13.75$& $-47.44<\fnl< 33.94$ &   36.8\\
BASS+MzLS (imag. cut)      & Nonlin. (Cons. II)& $ 25.03$& $ 29.12$& $  6.16<\fnl< 52.44$& $-14.22<\fnl< 80.54$ &   36.2\\
BASS+MzLS (comp. cut)     & Nonlin. (Cons. II)& $ 16.99$& $ 20.90$& $  0.26<\fnl< 41.76$& $-18.30<\fnl< 67.12$ &   35.8\\
DECaLS North                     & Nonlin. (Cons. II)& $ 41.02$& $ 44.89$& $ 23.33<\fnl< 66.78$& $  4.96<\fnl< 93.02$ &   41.1\\
DECaLS North                     & Nonlin. (Cons. II+CALIBZ+HI)& $ 55.46$& $ 60.44$& $ 36.78<\fnl< 84.05$& $ 17.86<\fnl<112.81$ &   38.4\\
DECaLS North                     & Nonlin. (Cons. II+nStar)& $ 31.45$& $ 34.78$& $ 14.14<\fnl< 55.79$& $ -5.81<\fnl< 80.80$ &   41.2\\
DECaLS North                     & Nonlin. (All Maps+nStar)& $  0.81$& $ -5.68$& $-29.73<\fnl< 16.71$& $-53.15<\fnl< 36.19$ &   45.1\\
DECaLS North (no DEC cut)      & Nonlin. (Cons. II)& $ 41.05$& $ 44.82$& $ 23.58<\fnl< 66.08$& $  6.40<\fnl< 91.42$ &   40.7\\
DECaLS North (imag. cut)   & Nonlin. (Cons. II)& $ 43.27$& $ 48.39$& $ 24.60<\fnl< 72.50$& $  4.71<\fnl<101.42$ &   35.1\\
DECaLS North (comp. cut)  & Nonlin. (Cons. II)& $ 40.55$& $ 44.63$& $ 22.41<\fnl< 67.11$& $  3.95<\fnl< 94.06$ &   41.4\\
DECaLS South                    & Nonlin. (Cons. II)& $ 31.24$& $ 33.21$& $ 14.89<\fnl< 52.40$& $ -5.11<\fnl< 74.35$ &   30.2\\
DECaLS South                    & Nonlin. (Cons. II+CALIBZ+HI)& $ 33.79$& $ 37.50$& $ 17.71<\fnl< 57.42$& $ -0.31<\fnl< 80.94$ &   30.8\\
DECaLS South                    & Nonlin. (Cons. II+nStar)& $ 14.34$& $  6.28$& $-21.19<\fnl< 30.01$& $-53.63<\fnl< 49.51$ &   31.9\\
DECaLS South                    & Nonlin. (All Maps+nStar)& $-36.76$& $-32.01$& $-49.38<\fnl<-13.61$& $-65.26<\fnl<  7.52$ &   31.5\\
DECaLS South (no DEC cut) & Nonlin. (Cons. II)& $ 43.79$& $ 46.79$& $ 30.16<\fnl< 63.41$& $ 16.38<\fnl< 82.72$ &   23.8\\
DECaLS South (imag. cut)        & Nonlin. (Cons. II)& $ 26.47$& $ 23.36$& $  3.18<\fnl< 47.84$& $-57.69<\fnl< 71.39$ &   30.0\\
DECaLS South (comp. cut)       & Nonlin. (Cons. II)& $ 29.62$& $ 31.76$& $ 13.00<\fnl< 51.58$& $ -9.78<\fnl< 74.28$ &   29.7\\
   \hline
    \end{tabular}
}
\end{table*}


\begin{itemize}
\item \textbf{Pixel completeness}
We remove pixels with low completeness from the DESI footprint by applying $f_{\rm pix}>0.5$, and find that the impact is negligible. Specifically, the cut removes \mr{$.6\%$} survey area and causes best fit $\fnl$ shifts only around $2\%$, from 28.58 to 28.07, see \textit{comp cut} Table \ref{tab:dr9method}. When investigated this impact on each region separately, BASS+MzLS increases around $10\%$, DECaLS North decreases $1\%$, and DECaLS South decreases around $5\%$.

\item \textbf{Imaging quality}
We remove pixels with poor imaging from the DESI footprint by applying the following cuts on imaging properties; $E[B-V]<0.1$, $nStar < 3000$, $depth_{g} > 23.2$, $depth_{r} > 22.6$, $depth_{z} > 22.5$, $psfsize_{g}<2.5$, $psfsize_{r}<2.5$, and $psfsize_{z}<2$. Overall the constraints are consistent despite best fit and marginalized mean estimates shift. Quantitatively, we lose about $8.2\%$ survey area, and the best fit $\fnl$ estimate changes about $2\%$ from $28.58$ to $29.16$. See \textit{imag cut} in Table \ref{tab:dr9method}. For BASS+MzLS only, the imaging cut increases the best fit by $62\%$ from $15.43$ to $25.03$. For DECaLS North and DECaLS South, the best fit increases by $5\%$ and $15\%$ respectively.

\item \textbf{Covariance}
We now use the mocks with $\fnl=76.92$ to construct a covariance matrix, and with the new covariance we observe a $12\%$ increase in the $\fnl$ constraint uncertainties and $11\%$ increase in the best fit estimate of $\fnl$.

\item \textbf{Lowest $\ell$} 
We decrease the largest mode (or increase the lowest $\ell$) used in estimating the best fit and $68\%$ confidence intervals. Fig. \ref{fig:mcmc_dr9elmin} illustrates the results for the DESI footprint and how they compared to BASS+MzLS, DECaLS North, or DECaLS South only results. Points represent marginalized mean estimates of $\fnl$ and errorbars represent $68$\% confidence from MCMC results.Overall we find that the constraints are robust against the largest mode.

\item \textbf{External maps} We also derive imaging weights using additional external maps for the neutral hydrogen column density (HI) and magnitude calibration errors in the z band (CALIBZ). With the new weights, we find the best fit estimates increase from $41.02$ to $55.46$ for DECaLS North and from $31.24$ to $33.79$ for DECaLS South.

\item \textbf{Declination cut} Our default analysis do not use the spurious islands in DECaLS North and DECaLS South below DEC = -30 to avoid potential calibration issues. \mr{PANASTARS are used for calibration below DEC of -30.} Without these cuts, best fit $\fnl$ estimates increase from $31.24$ to $43.79$ for DECaLS South and decrease from $41.02$ to $41.05$. This indicates that indeed there is an issue with DECaLS South below DEC of $-30$.

\end{itemize}

Overall we find that the declination cut is necessary for DECaLS South, while adding external templates for HI and CALIBZ, using a different covariance, or applying imaging and completeness cuts do not alter the constraints significantly.


\begin{figure}
    \centering
    \includegraphics[width=0.45\textwidth]{figures/mcmc_dr9regions.pdf} 
    \caption{Uncalibrated 2D constraints from the DR9 LRG sample for each imaging survey compared with that for the whole DESI footprint. The dark and light shades represent the $68\%$ and $95\%$ confidence intervals, respectively.}\label{fig:mcmc_dr9reg}
\end{figure}




\begin{figure}
    \centering
    \includegraphics[width=0.45\textwidth]{figures/fnl_elmin.pdf}     
    \caption{Robustness of the uncalibrated DR9 constraints w.r.t. the largest scale (lowest $\ell$ mode) used in MCMC regression. Points represent marginalized mean estimates of $\fnl$ and errorbars represent $68$\% confidence.}\label{fig:mcmc_dr9elmin}
\end{figure}