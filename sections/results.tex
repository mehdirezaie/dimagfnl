\section{Results}\label{sec:results}
\mr{This section presents our $\fnl$ constraints from the DESI LRG targets. The analysis is not carried out blindly. However, the cleaning methods are decided only based on the cross power spectrum and mean density contrast statistics.} 

\begin{figure*}
    \centering
    \includegraphics[width=0.9\textwidth]{figures/model_dr9.pdf} 
    \caption{The angular power spectrum of the DR9 LRG sample before (\textit{No weight}) and after correcting for imaging systematics using the linear and nonlinear methods with their corresponding best fit theory curves. The solid curve and grey shade respectively represent the mean power spectrum and $68\%$ error from the $\fnl=0$ mocks.}
    \label{fig:cl_dr9}
\end{figure*}

\subsection{DESI imaging LRG sample}
Figure \ref{fig:cl_dr9} shows the measured power spectrum of the DR9 LRG sample before and after applying imaging weights and the best fit theory curves. The solid line and the grey shade represent respectively the mean power spectrum and 1$\sigma$ error, estimated from the $\fnl=0$ lognormal simulations. The differences between various cleaning methods are significant on large scales ($\ell > 20$), but the small scale clustering measurements are consistent. By comparing \textit{linear two maps} to \textit{linear three maps}, we find that the measured clustering power on modes with $6\leq \ell < 10$ are noticeably different between the two methods. We associate the differences to the additional map for psfsize in the r-band, which is included in \textit{linear three maps}. On other scales, the differences between \textit{linear three maps} and \textit{linear eight maps} are negligible, supporting the idea that our feature selection procedure has been effective in identifying the primary maps which cause the large-scale excess clustering signal. Comparing \textit{nonlinear three maps} to \textit{linear three maps}, we find that the measured spectra on $4 \leq \ell < 6$ are very different, probably indicating some nonlinear spurious fluctuations with large scale characteristics due to extinction. Including stellar density in the nonlinear approach (\textit{nonlinear four maps}) further reduces the excess power relative to the mock power spectrum, in particular on modes between $2\leq \ell < 4$. When calibrated on the lognormal simulations, we find that these differences are recovered after accounting for over-correction. Therefore, we associate this subtraction to over fitting.


\subsubsection{Calibrated constraints}

\begin{figure}
    \raggedleft
    \includegraphics[width=0.424\textwidth]{mcmc_dr9methods1d.pdf}
    \includegraphics[width=0.45\textwidth]{figures/mcmc_dr9methods.pdf} 
    \caption{The calibrated constrains from the DR9 LRG targets. \textit{Top}: probability distribution for $\fnl$ marginalized over the shotnoise and bias. \textit{Bottom}: $68\%$ and $95\%$ probability distribution contours for the bias and $\fnl$ from the DR9 LRG sample before and after applying nonlinear cleaning methods. The lognormal mocks are used to calibrate these distributions for over correction.}\label{fig:mcmc_dr9}
\end{figure}

\begin{table*}
    \caption{The calibrated best fit, marginalized mean, and marginalized $68\%$ ($95\%$) confidence estimates for $\fnl$ from fitting the DR9 LRG power spectrum before and after correcting for imaging systematic effects.}
    \label{tab:dr9methodcalib}
   \centerline{%     
    \begin{tabular}{llllllll}
    \hline
    \hline
   &  & 	  & & $\fnl$ &  &  \\
   \cmidrule(r{.7cm}){3-6}
Footprint                               & Method & 	Best fit  & Mean & $ 68\%$ CL & $ 95\%$ CL & $\chi^{2}$/dof \\
    \hline
DESI                      & No Weight   & $113.18$& $115.49$& $ 98.14<\fnl<132.89$& $ 83.51<\fnl<151.59$ &   44.4/34\\
DESI                      & Nonlinear Three Maps& $ 47.38$& $ 48.81$& $ 36.08<\fnl< 61.44$& $ 25.03<\fnl< 75.64$ &   34.6/34\\
DESI                      & Nonlinear Four Maps& $ 48.92$& $ 50.10$& $ 36.88<\fnl< 63.31$& $ 24.87<\fnl< 77.78$ &   35.2/34\\
DESI                      & Nonlinear Nine Maps& $ 49.69$& $ 41.91$& $ 13.10<\fnl< 69.14$& $-15.96<\fnl< 91.84$ &   39.5/34\\
   \hline
    \end{tabular}
}
\end{table*}

All $\fnl$ constraints presented here are calibrated for the effect of over correction using the lognormal simulations. Table \ref{tab:dr9methodcalib} describes the best fit and marginalized mean estimates of $\fnl$ from fitting the power spectrum of the DR9 LRG sample before and after cleaning with the nonlinear approach given various combinations for the imaging systematic maps. Figure \ref{fig:mcmc_dr9} shows the marginalized probability distribution for $\fnl$ in the top panel, and the $68\%$ and $95\%$ probability contours for the linear bias parameter and $\fnl$ in the bottom panel, from our sample before and after applying various corrections for imaging systematics. Overall, we find the maximum likelihood estimates to be consistent among the various cleaning methods. We obtain $36.08 (25.03) < \fnl < 61.44(75.64)$ with $\chi^{2}=34.6$ for \textit{nonlinear three maps} over 34 degrees of freedom. Accounted for over-correction, we obtain $36.88(24.87) < \fnl < 63.31(77.78)$ with $\chi^{2}=35.2$ with the additional stellar density map in the \textit{nonlinear four maps}. With or without $nStar$, the confidence intervals are consistent with each other and more than $3\sigma$ off from zero PNG. Cleaning the sample with \textit{nonlinear nine maps} weakens our constraints to $13.10(-15.96) < \fnl < 69.14(91.84)$ with $\chi^{2}=39.5$. For comparison, we obtain $98.14(83.51) < \fnl < 132.89(151.59)$ at $68\% (95\%)$ confidence with $\chi^{2}=44.4$ for the \textit{no weight} approach. The uncalibrated probability contours are presented in Appendix \ref{sec:dr9uncalib}.


\subsubsection{Uncalibrated constraints: robustness tests}
Now we proceed to perform some robustness tests and assess how sensitive the $\fnl$ constraints are to the assumptions made in the analysis or the quality cuts applied to the data. For each case, we re-train the cleaning methods and derive new sets of imaging weights. Accordingly, for the cases where a new survey mask is applied to the data, we re-calculate the covariance matrices using the new survey mask to account for the changes in the survey window and integral constraint effects. Calibrating the mitigation biases for all of these experiments is beyond the scope of this work and redundant, as we are only interested in the relative shift in the $\fnl$ constraints after changing the assumptions. Therefore, the absolute scaling of the $\fnl$ constraints presented here are biased because of the over correction effect. Table \ref{tab:dr9method} summarizes the uncalibrated $\fnl$ constraints from the DR9 LRG sample. Our tests are as follows:

\begin{table*}
    \caption{The uncalibrated best fit and marginalized mean estimates for $\fnl$ from fitting the power spectrum of the DR9 LRG sample before and after correcting for systematics. The number of degrees of freedom is 34 (37 data points - 3 parameters). The lowest mode is $\ell=2$ and the covariance matrix is from the $\fnl=0$ clean mocks (no mitigation) except for the case with '+cov' in which the covariance matrix is from the $\fnl=76.9$ clean mocks (no mitigation).}
    \label{tab:dr9method}
   \centerline{%     
    \begin{tabular}{llllllll}
    \hline
    \hline
   &  & 	  & & $\fnl$ &  &  \\
   \cmidrule(r{.7cm}){3-6}
Footprint                               & Method & 	Best fit  & Mean & $ 68\%$ CL & $ 95\%$ CL & $\chi^{2}$ \\
    \hline
DESI                      & No Weight   & $113.18$& $115.49$& $ 98.14<\fnl<132.89$& $ 83.51<\fnl<151.59$ &   44.4\\
DESI                      & Linear Eight Maps& $ 36.05$& $ 37.72$& $ 26.13<\fnl< 49.21$& $ 16.31<\fnl< 62.31$ &   41.1\\
DESI                      & Linear Two Maps& $ 49.58$& $ 51.30$& $ 38.21<\fnl< 64.33$& $ 27.41<\fnl< 78.91$ &   38.8\\
DESI                      & Linear Three Maps& $ 36.63$& $ 38.11$& $ 26.32<\fnl< 49.86$& $ 16.36<\fnl< 63.12$ &   39.6\\
DESI                      & Nonlinear Three Maps& $ 28.58$& $ 29.79$& $ 18.91<\fnl< 40.59$& $  9.47<\fnl< 52.73$ &   34.6\\
DESI (imag. cut)          & Nonlinear Three Maps& $ 29.16$& $ 30.57$& $ 19.05<\fnl< 42.18$& $  9.01<\fnl< 54.81$ &   35.8\\
DESI (comp. cut)          & Nonlinear Three Maps& $ 28.07$& $ 29.48$& $ 18.38<\fnl< 40.50$& $  8.81<\fnl< 53.10$ &   34.5\\
DESI                      & Nonlinear Four Maps& $ 16.63$& $ 17.52$& $  7.51<\fnl< 27.53$& $ -1.59<\fnl< 38.49$ &   35.2\\
DESI                      & Nonlinear Nine Maps& $ -5.87$& $ -9.19$& $-21.45<\fnl<  2.40$& $-33.81<\fnl< 12.06$ &   39.5\\
DESI                     & Nonlinear Three Maps+$f_{\rm NL}=76.92$ Cov& $ 31.62$& $ 33.11$& $ 20.94<\fnl< 45.24$& $ 10.56<\fnl< 59.16$ &   33.5\\
\hline
BASS+MzLS                 & Nonlinear Three Maps& $ 15.43$& $ 19.01$& $ -1.17<\fnl< 39.43$& $-19.19<\fnl< 63.56$ &   35.6\\
BASS+MzLS                 & Nonlinear Four Maps& $ 13.12$& $ 15.39$& $ -4.59<\fnl< 35.56$& $-24.88<\fnl< 59.31$ &   34.7\\
BASS+MzLS                 & Nonlinear Nine Maps& $ -3.73$& $ -6.34$& $-27.11<\fnl< 13.75$& $-47.44<\fnl< 33.94$ &   36.8\\
BASS+MzLS (imag. cut)     & Nonlinear Three Maps& $ 25.03$& $ 29.12$& $  6.16<\fnl< 52.44$& $-14.22<\fnl< 80.54$ &   36.2\\
BASS+MzLS (comp. cut)     & Nonlinear Three Maps& $ 16.99$& $ 20.90$& $  0.26<\fnl< 41.76$& $-18.30<\fnl< 67.12$ &   35.8\\
DECaLS North              & Nonlinear Three Maps& $ 41.02$& $ 44.89$& $ 23.33<\fnl< 66.78$& $  4.96<\fnl< 93.02$ &   41.1\\
DECaLS North              & Nonlinear Four Maps& $ 31.45$& $ 34.78$& $ 14.14<\fnl< 55.79$& $ -5.81<\fnl< 80.80$ &   41.2\\
DECaLS North              & Nonlinear Five Maps& $ 55.46$& $ 60.44$& $ 36.78<\fnl< 84.05$& $ 17.86<\fnl<112.81$ &   38.4\\
DECaLS North              & Nonlinear Nine Maps& $  0.81$& $ -5.68$& $-29.73<\fnl< 16.71$& $-53.15<\fnl< 36.19$ &   45.1\\
DECaLS North (no DEC cut) & Nonlinear Three Maps& $ 41.05$& $ 44.82$& $ 23.58<\fnl< 66.08$& $  6.40<\fnl< 91.42$ &   40.7\\
DECaLS North (imag. cut)  & Nonlinear Three Maps& $ 43.27$& $ 48.39$& $ 24.60<\fnl< 72.50$& $  4.71<\fnl<101.42$ &   35.1\\
DECaLS North (comp. cut)  & Nonlinear Three Maps& $ 40.55$& $ 44.63$& $ 22.41<\fnl< 67.11$& $  3.95<\fnl< 94.06$ &   41.4\\
DECaLS South              & Nonlinear Three Maps& $ 31.24$& $ 33.21$& $ 14.89<\fnl< 52.40$& $ -5.11<\fnl< 74.35$ &   30.2\\
DECaLS South              & Nonlinear Four Maps& $ 14.34$& $  6.28$& $-21.19<\fnl< 30.01$& $-53.63<\fnl< 49.51$ &   31.9\\
DECaLS South              & Nonlinear Five Maps& $ 33.79$& $ 37.50$& $ 17.71<\fnl< 57.42$& $ -0.31<\fnl< 80.94$ &   30.8\\
DECaLS South              & Nonlinear Nine Maps& $-36.76$& $-32.01$& $-49.38<\fnl<-13.61$& $-65.26<\fnl<  7.52$ &   31.5\\
DECaLS South (no DEC cut) & Nonlinear Three Maps& $ 43.79$& $ 46.79$& $ 30.16<\fnl< 63.41$& $ 16.38<\fnl< 82.72$ &   23.8\\
DECaLS South (imag. cut)  & Nonlinear Three Maps& $ 26.47$& $ 23.36$& $  3.18<\fnl< 47.84$& $-57.69<\fnl< 71.39$ &   30.0\\
DECaLS South (comp. cut)  & Nonlinear Three Maps& $ 29.62$& $ 31.76$& $ 13.00<\fnl< 51.58$& $ -9.78<\fnl< 74.28$ &   29.7\\
   \hline
    \end{tabular}}
\end{table*}

\begin{figure}
    \centering
    \includegraphics[width=0.45\textwidth]{figures/mcmc_dr9regions.pdf} 
    \caption{The uncalibrated 2D constraints from the DR9 LRG sample for each imaging survey compared with that for the whole DESI footprint. The dark and light shades represent the $68\%$ and $95\%$ confidence intervals, respectively.}\label{fig:mcmc_dr9reg}
\end{figure}
\begin{figure*}
    \centering
    \includegraphics[width=0.85\textwidth]{figures/cldr9_lowell.pdf}
    \includegraphics[width=0.86\textwidth]{figures/fnl_elmin.pdf}  
    \caption{Top: The measured power spectra before and after \textit{nonlinear three maps}. The uncalibrated $\fnl$ constraints vs the lowest $\ell$ mode from the DR9 LRG sample cleaned with \textit{nonlinear three maps}. The points represent marginalized mean estimates of $\fnl$ and error bars represent $68$\% confidence.}\label{fig:mcmc_dr9elmin}
\end{figure*}

\begin{itemize}[itemindent=*]

\item \textbf{Linear methods}: We find consistent constraints from \textit{linear eight maps} and \textit{linear three maps} which suggests that not all imaging systematic maps are needed to completely mitigate systematic effects. We find $\sigma (\fnl) \sim 25$ for the linear methods. With the same set of imaging systematic maps, the nonlinear method yields a smaller constraint, $\Delta \fnl = -8$, and a better $\chi^{2}$ fit, e.g., $34.6$ vs $39.6$ for 34 degrees of freedom, even though the covariance matrix is fixed.

\item \textbf{Imaging regions}: We compare how our constraints from fitting the power spectrum of the whole DESI footprint compares to that from the power spectrum of each imaging region individually, namely BASS+MzLS, DECaLS North, and DECaLS South. Figure~\ref{fig:mcmc_dr9reg} shows the $68\%$ and $95\%$ probability contours on $\fnl$ and $b$ from each individual region, compared with that from DESI. The cleaning method here is \textit{nonlinear three maps}, and the covariance matrices are estimated from the $\fnl=0$ mocks. Overall, we find that the constraints from all imaging surveys are consistent with each other and DESI within $68\%$ confidence. Both BASS+MzLS and DECaLS South yield constraints consistent with $\fnl=0$ within $95\%$, but DECaLS North deviates from zero PNG at more than $2\sigma$. This motivates follow-up studies with the spectroscopic sample of LRGs in DECaLS North.

\item \textbf{Stellar density template (\textit{nStar})}: Adding the stellar density template (\textit{nonlinear four maps}) does not change the constraints from BASS+MzLS much, but it shifts the $\fnl$ distributions to lower values in DECaLS North and DECaLS South by $0.5\sigma$ and $\sigma$, respectively, reconciling all constraints with $\fnl=0$. We note that differences are more significant when all nine maps are used as input. This is somewhat expected as cleaning the data with more imaging systematic maps is more prone to the over-correction issue. \mr{We find that the shifts in $\fnl$ from adding $nStar$ are cancelled after accounting for the over correction bias. Comparing \textit{nonlinear four maps} and \textit{nonlinear three maps} in Table \ref{tab:dr9methodcalib} and \ref{tab:dr9method} is indicative that the $\fnl$ shifts after adding $nStar$ are indicative of over-correction due to correlations between the stellar density template and large-scale structure.} 

\item \textbf{Pixel completeness (\textit{comp. cut})}: We discard pixels with fractional completeness less than half to assess the effect of partially complete pixels on $\fnl$. This cut removes $0.6\%$ of the survey area, and no changes in the $\fnl$ constraints are observed.

\item \textbf{Imaging quality (\textit{imag. cut})}: Pixels with poor photometry are removed from our sample by applying the following cuts on imaging; $E[B-V]<0.1$, $nStar < 3000$, ${\rm depth}_{g} > 23.2$, ${\rm depth}_{r} > 22.6$, ${\rm depth}_{z} > 22.5$, ${\rm psfsize}_{g}<2.5$, ${\rm psfsize}_{r}<2.5$, and ${\rm psfsize}_{z}<2$. Although these cuts remove $8\%$ of the survey mask, there is a negligible impact on the best fit $\fnl$ from fitting the DESI power spectrum. However, when each region is fit individually, the BASS+MzLS constraint shift toward higher values of $\fnl$ by approximately $\Delta \fnl \sim 10$, whereas the constraints from DECaLS North and DECaLS South do not change significantly. 

\item \textbf{Covariance matrix (\textit{cov})}: We fit the power spectrum of our sample cleaned with \textit{nonlinear three maps} correction, but use the covariance matrix constructed from the $\fnl=76.92$ mocks. With the alternative covariance, a $12\%$ increase in the $\sigma \fnl$ is observed. We also find that the best fit and marginalized mean estimates of $\fnl$ increase by $10-11\%$. Overall, we find that the differences are not significant in comparison to the statistical precision.

\item \textbf{External maps (\textit{CALIBZ+HI})}: The neural network five maps correction includes the additional maps for HI and CALIBZ. With this correction, the best fit $\fnl$ increases from $41.02$ to $55.46$ for DECaLS North and from $31.24$ to $33.79$ for DECaLS South, which might suggest that adding HI and CALIBZ increases the input noise, and thus negatively impacts the performance of the neural network model. This test is not performed on BASS+MzLS due to a lack of coverage from the CALIBZ map. 

\item \textbf{Declination mask (\textit{no DEC cut})}: The fiducial mask removes the disconnected islands in DECaLS North and regions with DEC $<-30$ in DECaLS South, where there is a high likelihood of calibration issues as different standard stars are used for photometric calibrations. We analyze our sample without these cuts, and find that the best fit and marginalized $\fnl$ mean estimates from DECaLS South shift significantly to higher values of $\fnl$ by $\Delta \fnl \sim 10$, which supports the issue of photometric systematics in the DECaLS South region below DEC $=-30$. On the other hand, the constraints from DECaLS North do not change significantly, indicating the islands do not induce significant contaminations. \bbk{[Did you try cutting out more of DECaLS North, such as everything below the equator?]}

\item \textbf{Scale dependence (\textit{varying $\ell_{\rm min}$})}: We raise the value of the lowest harmonic mode $\ell_{\rm min}$ used for the likelihood evaluation during MCMC. This is equivalent to decreasing the highest scale of measurement in the power spectrum. By doing so, we anticipate a reduction in the impact of imaging systematics on $\fnl$ inference as lower $\ell$ modes are more likely to be contaminated. Figure \ref{fig:mcmc_dr9elmin} illustrates the power spectra before and after the correction with \textit{nonlinear three maps} in the top panel. The bottom panel shows the marginalized mean and $68\%$ error on $\fnl$ with \textit{nonlinear three maps} for the DESI, BASS+MzLS, DECaLS North, and DECaLS South regions. \mr{ALEX: We find that the mean estimates of $\fnl$ slightly shifts to higher values on scales $12<\ell<18$ in DECaLS North and BASS+MzLS when higher $\ell_{\rm min}$ is used. This is the opposite behavior from what one would expect if there were just a giant systematics induced spike at low $\ell$. So it shows that the issue here is more subtle than what one would have initially suspected.}

\end{itemize}

\subsection{Summary}
In summary, we find that the nonlinear methods outperform the linear methods in removing the excess clustering signal on large scales. Adding the stellar density map results in significant changes, however when accounted for the mitigation bias, all methods recover the same maximum likelihood estimate. With calibration on the lognormal mocks, the conservative approaches that use a small subset of imaging systematic maps show $\fnl$ detection at more than $2\sigma$ confidence. The most flexible nonlinear method with nine maps returns a bigger associated uncertainty which is consistent with $\fnl=0$. We also run various tests with cuts on the DR9 sample or changing the configuration or details of the analysis. Overall, we find consistent results across sub imaging surveys within DESI. However, our results show that a declination cut at DEC $=-30$ is necessary for DECaLS South to avoid potential calibration issues. Our analysis does not show a statistical demand for including external templates for HI and CALIBZ, using a different covariance matrix, or imposing additional cuts on the DR9 based on imaging and pixel completeness. We also obtain robust results regardless of the largest scale used for constraining $\fnl$.