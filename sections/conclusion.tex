\section{Discussion and Conclusion}\label{sec:conclusion}

We have measured the local PNG parameter $\fnl$ using the scale-dependent bias in the angular clustering of LRGs selected from the DESI Legacy Imaging Survey DR9. Our sample includes more than $12$ million LRG targets covering around $14,000$ square degrees in the redshift range of $0.2< z < 1.35$. We leverage early spectroscopy during DESI Survey Validation \citep{desi2023sv} to infer the redshift distribution of our sample (Figure \ref{fig:nz}). In our fiducial analysis, we have obtained a maximum likelihood value of $\fnl=47$ with a significant probability that $\fnl$ is greater than zero, $P(\fnl>0)=99.9$ (Figure \ref{fig:mcmc_dr9} and Table \ref{tab:dr9methodcalib}). 

The signature of local PNG is very sensitive to excess clustering signals caused by imaging systematic effects. We have applied a series of robustness tests to investigate the impact of how the galaxy selection function is determined. Specifically, both linear and nonlinear methods are applied using various combinations of imaging systematic maps (Galactic extinction, stellar density, depth in $grzW1$, psfsize in $grz$, and neutral hydrogen column density). We also examine the effect of different masks based on imaging. Overall, we find no change in the analysis that shifts the maximum likelihood value of $\fnl$ to a significantly lower value (Figure \ref{fig:mcmc_dr9reg}, Figure \ref{fig:mcmc_dr9elmin}, and Table \ref{tab:dr9method}). The only manner in which the significance of nonzero PNG decreases is due to the uncertainty on the measurement increasing when we employ more imaging systematic maps for the selection function estimation and by doing so remove large-scale clustering information (the effect of which on $\fnl$ recovery we have calibrated with mocks, as shown in Figure \ref{fig:fnlbias}).

\begin{figure}
    \centering
    \includegraphics[width=0.45\textwidth]{figures/fnl_history.pdf}
    \caption{History of constraints on local PNG $\fnl$ at $95\%$ confidence from single-tracer LSS \citep{slosar2008constraints,2013MNRAS.428.1116R, mueller2022primordial}, including our analysis with $25<\fnl<76$ (DESI photo LRG) and CMB surveys \citep{Komatsu_2003, Komatsu_2010, planck13, akrami2019planck}. The median $\fnl$ value is used in case the maximum likelihood estimate was not reported in the reference.}
    \label{fig:fnlhist}
\end{figure}

When comparing our fiducial results to recent CMB and QSO measurements, as shown in Figure \ref{fig:fnlhist}, we find a significant tension with CMB but consistent constraints with LSS within $95\%$ confidence. Either we have measured an $\fnl$ signal that is inconsistent with CMB measurements or there is a hidden source of systematic contamination in our data which cannot be mitigated with available imaging systematic maps. To be consistent with CMB measurements, our results would need to correspond to some kind of scale-dependent $\fnl$ model which has a larger non-gaussianity on LSS scales but negligible one at CMB scales \citep{sefusatti_2009, becker_2011}. Our analysis can be considered as the first attempt to identify major systematics in DESI, so we can be ready for constraining $\fnl$ with DESI spectroscopy. Internal DESI tests of the photometric calibration were unable to uncover DESI-specific issues, e.g., when comparing to Gaia data. The most significant trends that we find are with the E(B-V) map. The source of such a trend would be a mis-calibration of the E(B-V) map itself or the coefficients applied to obtain Galactic extinction corrected photometry. Such a mis-calibration would plausibly be proportional in amplitude to the estimated E(B-V) map, though it may not have E(B-V)’s spatial distribution. In order to explain the $\fnl$ signal we measure, such an effect would need to be approximately twice that of the trend we find with E(B-V). There are ongoing efforts within DESI to obtain improved Galactic extinction information, which will help establish if this is indeed the cause.