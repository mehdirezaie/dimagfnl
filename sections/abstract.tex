\begin{abstract}
We use angular clustering of luminous red galaxies from the Dark Energy Spectroscopic Instrument (DESI) imaging surveys to constrain the local primordial non-Gaussianity parameter $\fnl$. Our sample comprises over 12 million targets, covering 14,000 square degrees of the sky, with redshifts in the range $0.2< z < 1.35$. We identify Galactic extinction, survey depth, and astronomical seeing as the primary sources of systematic error, and employ linear regression and artificial neural networks to alleviate non-cosmological excess clustering on large scales. Our methods are tested against log-normal simulations with and without $\fnl$ and systematics, showing superior performance of the neural network treatment in reducing remaining systematics. Assuming the universality relation, we find \mout{$\fnl = 47^{+14(+29)}_{-11(-22)}$} \mr{$\fnl = 34^{+24(+50)}_{-44(-73)}$} at 68\%(95\%) confidence. \mout{With a more aggressive treatment, including regression against the full set of imaging maps, our maximum likelihood value shifts slightly to $\fnl \sim 50$ and the uncertainty on $\fnl$ increases due to the removal of large-scale clustering information.} We apply a series of robustness tests (e.g., cuts on imaging, declination, or scales used) that show consistency in the obtained constraints. \mr{With a more conservative treatment that involves regression against three imaging maps, we obtain $\fnl = 46^{+15(+30)}_{-13(-25)}$.} \mout{Despite extensive efforts to mitigate systematics, our measurements indicate $\fnl > 0$ with a $99.9$ percent confidence level. This outcome raises concerns as it could be attributed to unforeseen systematics, including calibration errors or uncertainties associated with low-$\ell$ systematics in the extinction template. Alternatively, it could suggest a scale-dependent $\fnl$ model--causing significant non-Gaussianity around large-scale structure while leaving cosmic microwave background scales unaffected. Our results encourage further studies of $\fnl$ with DESI spectroscopic samples, where the inclusion of 3D clustering modes should help separate imaging systematics.} \mr{As mitigating with more maps removes large-scale clustering information, our results encourage better systematic treatment methods that avoid over-correction and preserve the constraining power of data. This work can be considered as the first attempt to identify major systematics in the DESI LRG sample, so we can be ready to constrain $\fnl$ with DESI Spectroscopy.}
\end{abstract}

% \mr{The corresponding large-scale power may stem from unforeseen systematics, including calibration errors or uncertainties associated with low-$\ell$ systematics in the extinction template. The only approaches that yield results consistent with $\fnl=0$ are those that diminish the constraining power of the dataset. For instance, we obtain $\fnl = 34^{+24(+50)}_{-44(-73)}$ using a more aggressive treatment that involves regression against the full set of imaging maps.}