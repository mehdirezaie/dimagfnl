\begin{abstract}
\mr{We use} the angular power spectrum of luminous red galaxies (LRGs) selected from the Dark Energy Spectroscopic Instrument (DESI) imaging surveys to constrain the local primordial non-Gaussianity parameter $\fnl$. Our sample comprises over $12$ million LRG targets, spanning approximately $14,000$ square degrees of the sky, with redshifts \mout{ranging from} \mr{in the range} $0.2< z < 1.35$. Galactic extinction, survey depth, and astronomical seeing are identified as the primary sources of systematic error using feature selection and cross-correlation techniques. Linear regression and artificial neural networks are applied to mitigate systematics and alleviate \mr{non-cosmological} excess clustering signals on large scales. Our treatment methods are tested and calibrated rigorously against log-normal density simulations with and without $\fnl$ and systematic effects. We find that the neural network treatment outperforms linear regression in reducing remaining systematics in the DESI LRG sample. Assuming the universality relation, we find $\fnl = 47^{+14~(+29)}_{-11~(-22)}$ at $68\%$~($95\%$) confidence \mr{for our fiducial method}. Applying a more aggressive systematics treatment that includes regression against the full set of imaging maps we identified, our maximum likelihood value changes only slightly to $\fnl \sim 50$, but the uncertainty on $\fnl$ increases due to the aggressive treatment removing large-scale clustering information. We apply a series of robustness tests (e.g., cuts on imaging, declination, or scales \mr{used}) that show \mout{remarkable} consistency in the obtained constraints. \mr{Despite various attempts to mitigate systematics, we have measured $\fnl > 0$ with a $99.9$ percent confidence level. This result raises concerns as it could be attributed to unforeseen systematics, such as calibration errors in photometric zero-point determination or Galactic extinction corrections. One particular culprit might be the uncertainties associated with low-$\ell$ systematics in the extinction template. Alternatively, it may indicate the presence of a new cosmological signal, specifically a scale-dependent $\fnl$ model that induces significant non-Gaussianity around large-scale structure while leaving cosmic microwave background scales unaffected.} \mout{Our fiducial result can either be interpreted as a strong detection of non-zero $\fnl$ with the probability $P(\fnl>0)=99.9$ per cent, inconsistent with what is measured from Planck, or evidence for unknown sources of variation in the observed galaxy density (e.g., from calibration errors in photometric zero-point determination or in Galactic extinction corrections).} Our results motivate follow-up studies of $\fnl$ with DESI spectroscopic samples, where the inclusion of 3D clustering modes should help separate imaging systematics.
\end{abstract}