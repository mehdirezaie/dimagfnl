\begin{abstract}
We use angular clustering of luminous red galaxies from the Dark Energy Spectroscopic Instrument (DESI) imaging surveys to constrain the local primordial non-Gaussianity parameter $\fnl$. Our sample comprises over 12 million targets, covering 14,000 square degrees of the sky, with redshifts in the range $0.2< z < 1.35$. We identify Galactic extinction, survey depth, and astronomical seeing as the primary sources of systematic error, and employ \mout{linear regression and} artificial neural networks to alleviate non-cosmological excess clustering on large scales. Our methods are tested against log-normal simulations with and without $\fnl$ and systematics, \mr{to characterize the impact of the systematics mitigation procedure on large-scale clustering signal}\mout{showing superior performance of the neural network treatment in reducing remaining systematics}. Assuming the universality relation, we find \mout{$\fnl = 47^{+14(+29)}_{-11(-22)}$} \mr{$\fnl = 34^{+24(+50)}_{-44(-73)}$} at 68\%(95\%) confidence \mr{for our fiducial analysis. This uses regression against the full set of nine imaging property maps we consider.} \mout{With a more aggressive treatment, including regression against the full set of imaging maps, our maximum likelihood value shifts slightly to $\fnl \sim 50$ and the uncertainty on $\fnl$ increases due to the removal of large-scale clustering information.} We apply a series of robustness tests (e.g., cuts on imaging, declination, or scales used) that show consistency in the obtained constraints. \mr{An important aspect of this work is to properly account for how the regression method biases the measured angular power-spectrum and degrades the $\fnl$ constraining power. The use of the full nine maps more than doubles the uncertainty compared to using only the three most significant maps in the regression.} \mout{Despite extensive efforts to mitigate systematics, our measurements indicate $\fnl > 0$ with a $99.9$ percent confidence level. This outcome raises concerns as it could be attributed to unforeseen systematics, including calibration errors or uncertainties associated with low-$\ell$ systematics in the extinction template. Alternatively, it could suggest a scale-dependent $\fnl$ model--causing significant non-Gaussianity around large-scale structure while leaving cosmic microwave background scales unaffected.} \mr{Our results thus motivate better systematic treatment methods that avoid over-correction to preserve large-scale clustering information. Additionally,} our results encourage further studies of $\fnl$ with DESI spectroscopic samples, where the inclusion of 3D clustering modes should help separate imaging systematics \mr{and lessen the degradation in the $\fnl$ uncertainty}.
\end{abstract}