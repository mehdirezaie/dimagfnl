\begin{abstract}
We use angular clustering of luminous red galaxies from the Dark Energy Spectroscopic Instrument (DESI) imaging surveys to constrain the local primordial non-Gaussianity parameter $\fnl$. Our sample comprises over 12 million targets, covering 14,000 square degrees of the sky, with redshifts in the range $0.2< z < 1.35$. We identify Galactic extinction, survey depth, and astronomical seeing as the primary sources of systematic error, and employ linear regression and artificial neural networks to alleviate non-cosmological excess clustering on large scales. Our methods are tested against simulations with and without $\fnl$ and systematics, showing superior performance of the neural network treatment. The neural network with a set of nine imaging property maps passes our systematic null test criteria, and is chosen as the fiducial treatment. Assuming the universality relation, we find $\fnl = 34^{+24(+50)}_{-44(-73)}$ at 68\%(95\%) confidence. We apply a series of robustness tests (e.g., cuts on imaging, declination, or scales used) that show consistency in the obtained constraints. We study how the regression method biases the measured angular power-spectrum and degrades the $\fnl$ constraining power. The use of the nine maps more than doubles the uncertainty compared to using only the three primary maps in the regression. Our results thus motivate the development of more efficient methods that avoid over-correction, protect large-scale clustering information, and preserve constraining power. Additionally, our results encourage further studies of $\fnl$ with DESI spectroscopic samples, where the inclusion of 3D clustering modes should help separate imaging systematics and lessen the degradation in the $\fnl$ uncertainty.
\end{abstract}