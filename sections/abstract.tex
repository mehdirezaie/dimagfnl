\begin{abstract}
This paper uses the angular power spectrum of luminous red galaxies (LRGs) selected from the Dark Energy Spectroscopic Instrument (DESI) imaging surveys to constrain the local primordial non-Gaussianity parameter $\fnl$. Our dataset comprises over $14$ million LRG targets, spanning approximately $18,000$ square degrees of the sky, with redshifts ranging from $0.2< z < 1.35$. The FFTLog algorithm is employed to enable modeling the power spectrum on large scales. Galactic extinction, survey depth, and astronomical seeing are identified as the primary sources of systematic error using feature selection and cross-correlation techniques. Artificial neural networks are applied to alleviate excess clustering signals on large scales. Our treatment methods are tested and calibrated rigorously against lognormal density simulations with $\fnl=0$ and $76.9$, and systematic effects included. Assuming halo bias only depends on mass, we find $36.08~(25.03) < \fnl < 61.44~(75.64)$ at $68\%$~($95\%$) confidence. Our constraints weaken to $13.10~(-15.96) < \fnl < 69.14~(91.84)$ if the spurious fluctuations in the galaxy density field are regressed out against all available imaging systematic maps. We also test the robustness of the $\fnl$ constraints against different assumptions and variations in the data analysis pipeline, only finding significant photometric calibration issues in the South Galactic Cap below declination of $-30$. Assuming Planck's constraint on the value of $\fnl$ is accurate a priori, our best fit estimates of $\fnl \sim 47-50$ provide evidence for some unknown calibration issues which cannot be addressed with available imaging maps. Our results show that constraining $\fnl$ requires a careful analysis of imaging systematic effects and a thorough understanding of astrophysical factors that impact the measured clustering signal on large scales. \mr{While more theoretical work and simulations are required to better understand halo assembly bias, our results motivate follow-up studies of $\fnl$ with DESI spectroscopic samples, which are less prone to contamination issues and provide a more uniform selection of galaxies.}
\end{abstract}