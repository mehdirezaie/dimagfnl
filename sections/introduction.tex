\section{Introduction}
\label{sec:introduction}

Inflation is a widely accepted paradigm in modern cosmology that explains many important characteristics of our Universe. It predicts that the early Universe underwent a period of accelerated expansion, resulting in the observed homogeneity and isotropy of the Universe on large scales \citep[see, e.g.,][for a review]{weinberg2013observational}. After the period of inflation, the Universe entered a phase of reheating in which primordial perturbations were generated, setting the initial seeds for structure formation \citep{kofman1994reheating, bassett2006inflation, lyth2009primordial}. Although inflation is widely accepted as a convincing explanation, the characteristics of the field or fields that drive the inflationary expansion remain largely unknown in contemporary observational cosmology. While early studies of the cosmic microwave background (CMB) and large-scale structure (LSS) suggested that primordial fluctuations are both Gaussian and scale-invariant \citep{PhysRevD.69.103501, guth2005inflationary}, some alternative classes of inflationary models predict deviations from these predictions, including some levels of non-Gaussianities in the primordial gravitational field. Non-Gaussianities are a measure of the degree to which the distribution of matter in the Universe deviates from a Gaussian distribution, which would have important implications for our understanding of the early Universe and the formation of structure \citep[see, e.g.,][]{2010CQGra..27l4011D, Biagetti2019Galax...7...71B}.

In its simplest form, primordial non-Gaussianity (PNG) depends on the local value of the primordial potential $\phi$ \citep{komatsu2001acoustic},
\begin{equation}
 \Phi = \phi + \fnl [\phi^{2} - <\phi^{2}>],
\end{equation}
where $\fnl$ is a nonlinear coupling constant and $\phi$ is assumed to be a Gaussian random field. Local-type PNG generates a primordial bispectrum, which peaks in the squeezed limit where one of the three wave vectors is much smaller than the other two. This means that one of the modes is on a much larger scale than the other two, and this mode couples with the other two modes to generate a non-Gaussian signal, which then affects the local number density of galaxies. This coupling induces a distinct scale-dependent bias in the galaxy distribution, which can be detected and measured through two-point clustering statistics \citep{dalal2008imprints}.

Different inflationary models involve different assumptions about the physics that drove the early expansion, and as a result, they predict different values for $\fnl$. For instance, the standard single-field slow-roll inflationary model predicts a small value of $\fnl \sim 0.01$ \citep[see, also,][for a discussion]{2003JHEP...05..013M}. On the other hand, some alternative inflationary scenarios involve multiple scalar fields that can interact with each other during inflation, leading to the generation of larger levels of non-Gaussianities. These models predict considerably larger values of $\fnl$ that can reach up to $100$ or higher \citep[see, e.g.,][]{alvarez2014arXiv1412.4671A, de2017next}. Obtaining accurate and robust constraints on $\fnl$ is therefore crucial in advancing our understanding of the dynamics of the early Universe. With $\sigma (\fnl)\sim 1$, we can rule out or confirm specific models of inflation and gain insight into the physics that drove the inflationary expansion.

The current tightest bound on $\fnl$ comes from Planck's bispectrum measurement of CMB anisotropies, $\fnl=0.9\pm 5.0$ \citep{akrami2019planck}. Limited by cosmic variance, CMB data alone cannot enhance statistical precision of $\fnl$ measurements enough to break the degeneracy amongst various inflationary paradigms \citep[see, e.g.,][]{ade2019simons}. However, by combining LSS with CMB data, we can partially or even completely cancel out the effects of cosmic variance, and thus significantly reduce statistical uncertainties down to $\sigma(\fnl)\sim 1$ \citep[see, e.g.,][]{schmittfull2018PhRvD}. This is because LSS provides a measurement of the same underlying signal, but from a different vantage point and at a later time in Universe history. Also, constraining $\fnl$ with the three-point clustering of LSS is hindered by nonlinearities raised from structure formation, which is non-trivial to model and disentangle from the primordial signal \citep{baldauf2011galaxy, baldauf2011primordial}. As a novel approach, UV Luminosity Function probes galaxy abundances and structure formation on small scales (e.g., $k \sim 2~{\rm Mpc}^{-1}$), which are otherwise impossible to explore with the scale-dependent bias. \cite{sabti2021JCAP} used UV Luminosity Function from the Hubble Space Telescope catalogues \citep{bouwens2015ApJ} to find $-166<\fnl<497$ at $95\%$ confidence. Even though this approach is still not competitive with the current bounds from CMB and LSS, upcoming surveys such as the James Webb Space Telescope and the Nancy Grace Roman Space Telescope are forecast to yield up to four times improvements on $
\fnl$ constraints from UV Luminosity Function. 


Currently, the most precise constraints on $\fnl$ from LSS reach a level of $\sigma(\fnl) \sim 20-30$, with the majority of the constraining power coming from the two-point statistics that utilize the scale-dependent bias \citep{2019JCAP...09..010C, mueller2022primordial, 2022PhRvD.106d3506C, 2022arXiv220111518D}. However, further work is needed to fully harness the potential of the scale-dependent bias in constraining $\fnl$. Specifically, more simulation-based studies of halo-assembly bias are necessary to properly model the galaxy power spectrum in the presence of local PNG \citep{2020JCAP...12..013B, 2020JCAP...12..031B, 2022JCAP...11..013B, 2023JCAP...01..023L}. In addition to the theoretical challenges, measuring $\fnl$ with the scale-dependent bias effect is nonetheless incredibly challenging due to various observational systematic effects that modulate the galaxy power spectrum on large scales, where the primordial signal is sensitive. For instance, survey geometry mixes clustering power on different angular scales \citep{beutler2014clustering,wilson2017rapid}. Relativistic effects also generate PNG-like scale-dependent signatures on large scales, which interfere with measuring $\fnl$ with the scale-dependent bias effect \citep{wang2020}. Similarly, matter density fluctuations with wavelengths larger than survey size, known as super-sample modes, modulate the galaxy power spectrum \citep{castorina2020JCAP}. In a similar way, the peculiar motion of the observer can mimic a PNG-like scale-dependent signature through aberration, magnification and the Kaiser-Rocket effect, i.e., a systematic dipolar apparent blue-shifting in the direction of the observer's peculiar motion \citep{2021JCAP...11..027B}. However, acting radially, this effect is subdominant in angular clustering measurements. An additional potential cause of systematic error arises from the fact that the mean galaxy density used to construct the density contrast field is estimated from the available data, rather than being known a priori. This introduces what is known as an integral constraint effect, which can cause the power spectrum on modes near the size of the survey to be artificially suppressed, effectively pushing it towards zero \citep{peacock1991large,de2019integral}. Accounting for these theoretical systematic effects in the galaxy power spectrum is crucial to obtain unbiased inference \citep[see, e.g.,][]{riquelme2022primordial}. 

Other observational systematics are driven by varying imaging conditions across the sky \citep{ross2011} and photometric calibration issues in data reduction that appear as spurious fluctuations in the observed density field of galaxies \citep{huterer2013calibration}. Compared to the integral constraint and survey geometry effects, this particular type of systematic error is much more complex to deal with, and has proven to be a significant obstacle in past efforts to study local primordial non-Gaussianity using the scale-dependent bias \citep[see, e.g.,][]{pullen2013systematic, Ho2015JCAP...05..040H}. It is anticipated that imaging systematic errors will be particularly problematic for wide-area galaxy surveys that observe regions of the night sky closer to the Galactic plane and that seek to incorporate more lenient selection criteria in order to include fainter galaxies \citep[see, e.g,][]{kitanidis2020imaging}. As an ongoing wide-area galaxy survey, the Dark Energy Spectroscopic Instrument (DESI) uses $5000$ robotically-driven fibers to simultaneously collect spectra. DESI is designed to deliver an unparalleled volume of spectroscopic data over its five year mission that promises to deepen our understanding of the equation of state for dark energy \citep{aghamousa2016desi}. If systematic errors can be effectively controlled, DESI has the potential to offer competitive constraints on local PNG, with an anticipated precision of $\sigma(\fnl)=5$ \citep{aghamousa2016desi}. By combining DESI data with those of other forthcoming surveys, such as the Rubin Observatory and SphereX, it may be possible to achieve an unparalleled level of precision, potentially reaching $\sigma (\fnl) \sim 1$. \citep[see, e.g.,][]{Heinrich2022AAS...24020203H}.
 
The primary objective of this paper is to utilize the scale-dependent bias signature in the angular power spectrum of luminous red galaxies selected from DESI imaging data to constrain the value of $\fnl$. By prioritizing the treatment of imaging systematic errors, we aim to lay the groundwork for subsequent studies of local PNG with DESI spectroscopy. To prepare our sample for such a delicate signal, we employ linear multivariate regression and neural networks to clean and refine the data from spurious density fluctuations caused by various imaging conditions. Our findings are validated using suites of lognormal simulations featuring values of $\fnl=0$ and $76.9$. With cross correlation techniques, we thoroughly characterize and examine potential sources of systematic error, including survey depth, astronomical seeing, photometric calibration, Galactic extinction, and local stellar density.

This paper is structured as follows. Section \ref{sec:data} describes the galaxy sample from DESI imaging and lognormal simulations with, or without, PNG and synthetic systematic effects. Section \ref{sec:method} outlines the theoretical framework for modeling the angular power spectrum, strategies for handling various observational and theoretical systematic effects, and statistical techniques for measuring the significance of remaining systematics in our sample after mitigation. Our results are presented in Section \ref{sec:results}, and Section \ref{sec:conclusion} summarizes our conclusions and directions for future work.