\section{Introduction}
\label{sec:introduction}
Current observations of the cosmic microwave background (CMB), large-scale structure (LSS), and supernovae (SN) are explained by a cosmological model that consists of dark energy, dark matter, and ordinary luminous matter, which has gone through a phase of rapid expansion, known as \textit{inflation},  at its early stages \citep[see, e.g.,][]{weinberg2013observational}. The theory of inflation elegantly addresses fundamental issues with the hot Big Bang theory, such as the isotropy of the CMB temperature, absence of magnetic monopole, and flatness of the Universe \mr{(see, e.g., Ryden 2002)}. At the end of inflation, the Universe was reheated and primordial fluctuations are generated to seed the subsequent growth of structure \mr{(Kofman et al 1994, Bassett et al 2006, Lyth and Liddle 2009)}. While the presence of an inflationary era is certain but the details of the inflation field still remain highly unknown, and statistical properties of primordial fluctuations pose as one of the puzzling questions in modern observational cosmology. Analyses of cosmological data have revealed that initial conditions of the Universe are consistent with Gaussian fluctuations \mr{(Guth and Kaiser 2005)}; however, there are some classes of models that predict some levels of non-Gaussianities in the primordial gravitational field. In its simplest form, primordial non-Gaussianity depends on the local value of the gravitational potential $\phi$ and is parameterized by a nonlinear parameter $\fnl$ \citep{komatsu2001acoustic},
\begin{equation}
    \Phi = \phi + \fnl [\phi^{2} -  <\phi^{2}>].
\end{equation}
Standard slow roll inflation predicts $\fnl$ to be of order $10^{-2}$, while multifield theories predict considerably higher values than unity \mr{(see, e.g., Putter et al 2017)}. Therefore, a robust measurement of $\fnl$ can be considered as the first stepping stone toward better understanding the physics of the early Universe.  PNG alters local number density of galaxies by coupling the long and small wavelength modes of dark matter gravitational field, and as a result it introduces a scale-dependent shift in halo bias \citep[see, e.g.,][]{dalal2008imprints, slosar2008constraints},
\begin{equation}
\Delta b \sim \fnl \frac{(b - p)}{k^{2}},
\end{equation}
where p determines the response of the tracer to the halo gravitational field. Assuming universality of the halo function, i.e., the occupation of halos can be determined from the mass, $p=1$. However, numerical simulations have shown the halo mass function of tracers that are result of recent mergers could depend on more parameters other than mass, and thus $p=1.6$.  Because of the $k^{-2}$ dependence, the effect of local primordial non-Gaussianity leaves its signature on the large scales in the two-point clustering of large-scale structure. 

Current tightest bound on $\fnl$ comes from the three-point clustering analysis of the CMB temperature anisotropies by the Planck satellite, $\fnl=0.9\pm 5.0$ \citep{akrami2019planck}. Next generations of CMB experiments, will improve this constraint but since CMB is limited by cosmic variance, it alone cannot further enhance to break the degeneracy amongst inflationary models \mr{(see, e.g., Ade et al 2019)}. However, combining CMB with LSS data could cancel cosmic variance, partially if not completely, and enhance these limits to a precision level required to differentiate between various inflationary models \citep[see, e.g.,][]{schmittfull2018PhRvD}.  Constraining $\fnl$ with the three point clustering of LSS is also hindered by the late-time nonlinear effects raised due to structure growth \mr{(see, e.g., Baldauf et al 2011)}, and this leaves the scale-dependent bias effect a smoking gun for constraining local PNG with LSS. 

Measuring $\fnl$ using the scale-dependent bias is however very challenging due to the presence of systematic effects which cause excess clustering signal on the same scales sensitive to $\fnl$. These systematics can be broadly classified into theoretical and observational. Major theoretical systematic effects are caused by the geometry of survey -- a fact that we never observe the full night sky -- which results in coupling different angular modes \mr{(see, e.g., Beutler et al. 2014, de Mattia and Ruhlmann-Kleider 2019)}. The other effect is commonly referred to as integral constraint and is raised due to our estimation of the mean density directly from data itself, which pushes the clustering signal on modes near the size of survey to zero \mr{(Peacock and Nicholson 1991, Wilson et al 2015)}. Ignoring any of these effects leads to biased $\fnl$ constraints \mr{(see, e.g., Riquelme et al 2022)}. On the other hand, observational systematics are primarily caused by varying imaging properties across the sky or calibration issues which leave spurious fluctuations in target density field \mr{(see, e.g., Huterer et al 2013)}. This type of systematic error is much more difficult to handle and has hindered previous studies of local PNG with galaxy and quasar clustering \citep[see, e.g.,][]{Ho2015JCAP...05..040H}. For instance, \cite{pullen2013systematic} found that the level of systematic contamination in the quasar sample of SDSS \mr{DRX} does not allow a robust $\fnl$ measurement. These imaging systematic issues are expected to be severe for wide-area galaxy surveys that observe the night sky closer to the Galactic plane and attempt to loosen the selection criteria to incorporate fainter targets. Beside canceling cosmic variance, cross correlating different tracers is a technique to alleviate systematic error, as each tracer might respond differently to a source of systematics. \mr{Giannantonio et al 2014} presents constraints using the integrated Sachs Wolfe effect. \mr{McCarthy et al (2022)} uses cosmic infrared background as a proxy for halos and cosmic microwave background lensing as a proxy for matter, finding no evidence for local primordial non-Gaussianity. 


 DESI utilizes robots to collect $5000$ spectra simultanously, and it is going to deliver an unparalleled amount of spectroscopic data up to redshift \mr{X}, which will complete our understanding of the energy contents of the Universe. With the volume probed and assuming imaging systematics are under control, DESI along with other upcoming surveys such as Rubin Observatory, and SphereX are forecast to yield unprecedented constraints on $\fnl$ as well \mr{(Abell et al 2009, Dore et al 2014, Aghamousa et al 2016)}. DESI preselects its targets from its dedicated imaging surveys, as known as Legacy Surveys, which are collected between 2014 and 2019 from three ground-based telescopes in Chile and the US. Understanding and calibration of systematic error in DESI imaging data is of paramount importance since these effects from imaging catalogs could potentially be inherited into spectroscopic catalogs, and thus impact the science one can do with DESI data. DESI targets galaxies and quasars and the effect of observational systematics in the DESI imaging data have been studied in great detail in \mr{Kitanidis et al (???), Rezaie et al (2021), Zhou et al (2021), and Chaussidon et al (2022)}. Improving methods to characterize systematic error in these samples is also important for measuring $\fnl$. We have a lot of amazing methods to eliminate the effect of imaging systematics. Some of these methods are based on cross correlating the map of target density with maps for imaging properties, while the other methods use a regression analysis to regress out the modes of imaging properties from the target density. While these methods are essentially the same, but they have their own limitations and constraints. For instance the cross correlation techniques can only be used for 2D data and could be time consuming for a large survey, but the template-based regression methods could be fast but yield biased results or remove some of the true clustering signal. Specifically related to the regression based methods, there is a little effort to calibrate and characterize the amount of the true clustering signal which is removed during the cleaning process. For studies like BAO and RSD, these effect might not matter \mr{(see, e.g., Merz et al 2021)}, however, as these effects are very prominent on large scales \mr{(see, e.g., Rezaie et al 2020, Mueller et al 2022)}, they could introduce biases in $\fnl$ constraints.  
 
 With the high importance of systematic error, the primary focus of this paper is to present an exquisite study of imaging systematic error and characterization of mitigation biases for measuring $\fnl$. We also present enhanced statistical tools to address the data quality and significance of residual systematic error. In this paper, we use the photometric sample of luminous red galaxies from the DESI Legacy Imaging Surveys Data Release 9, hereafter referred to as DR9, to constrain the local primordial non-Gaussianity parameter $\fnl$, while testing the robustness of our results against various sources of systematic effects. We also make use of spectroscopic data from DESI Survey Validation to determine the redshift distribution of galaxies. We cross correlate the DR9 density field with the templates of imaging realities to assess the effectiveness of treatment methods and to characterize the significance of residual systematic error. Section \ref{sec:data} describes the DR9 sample and simulations with and without PNG and imaging systematic effects, and Section \ref{sec:method} outlines the theory for modeling angular power spectrum and analysis techniques for quantifying various observational systematic effects. Finally, we present the results in Section \ref{sec:results}, and conclude with a comparison to previous $\fnl$ constraints in Section \ref{sec:conclusion}.