\section{Analysis techniques}
\label{sec:method} 
\mr{Ashley: I think somewhere in here (probably near the beginning) you need to describe how the 3 regions are combined, since I think you model their selection function separately and then combine for almost all results shown.} 
\mr{Unless it is specified, we measure auto power spectrum, cross power spectrum, and mean density contrast for the entire footprint. However, the imaging weights are derived by regressing the observed density vs imaging systematic maps in each footprint separately.}





\subsection{Power spectrum estimator}
The pseudo angular power spectrum \citep{hivon2002master} is utilized to extract information from the galaxy density contrast field, $\delta_{g}$, 
\begin{align}\label{eq:delta}
    \delta_{g} &= \frac{\rho- \overline{\rho}}{\overline{\rho}},
\end{align}
by decomposing it into spherical harmonics, $Y_{\ell m}$,
\begin{equation}
        a_{\ell m} = \int d\Omega ~ \delta_{g} W Y^{*}_{\ell m}.
\end{equation}
The mean galaxy density $\overline{\rho}$ is estimated from the entire LRG sample\footnote{The mean galaxy density is calculated separately for each region when we fit the power spectrum from each region individually.} and \mr{the survey window $W$ is determined by comparing the number of randoms to the expected number}. Then, the angular power spectrum is estimated by
\begin{equation}\label{eq:pusedocell}
        \tilde{C}_{\ell} = \frac{1}{2\ell +1} \sum_{m=-\ell}^{\ell} |a_{\ell m}|^{2}.
\end{equation}

We use the implementation of \texttt{anafast} from the \textsc{HEALPix} package \citep{gorski2005healpix} to do fast harmonic transforms and estimate the pseudo angular power spectrum and cross power spectrum. When the sky coverage of a survey is incomplete, this estimator yields a biased power spectrum. Specifically, the survey mask causes correlations between different harmonic modes \citep{beutler2014clustering,wilson2017rapid}, \mr{and the measured clustering power is smoothed on scales near the survey size.} Since these scales are highly sensitive to local primordial non-Gaussianity, it is crucial to account for these systematic effects in the model galaxy power spectrum to obtain unbiased $\fnl$ constraints, \mr{which we describe in the next subsection}.

 \subsection{Modelling}
 
Relativistic effects also generate PNG-like scale-dependent signatures on large scales, which interfere with measuring $\fnl$ with the scale-dependent bias effect \citep{wang2020}. Similarly, matter density fluctuations with wavelengths larger than survey size, known as super-sample modes, modulate the galaxy power spectrum \citep{castorina2020JCAP}. In a similar way, the peculiar motion of the observer can mimic a PNG-like scale-dependent signature through aberration, magnification and the Kaiser-Rocket effect, i.e., a systematic dipolar apparent blue-shifting in the direction of the observer's peculiar motion \citep{2021JCAP...11..027B}. However, acting radially, this effect is subdominant in angular clustering measurements. An additional potential cause of systematic error arises from the fact that the mean galaxy density used to construct the density contrast field is estimated from the available data, rather than being known a priori. This introduces what is known as an integral constraint effect, which can cause the power spectrum on modes near the size of the survey to be artificially suppressed, effectively pushing it towards zero \citep{peacock1991large,de2019integral}. Accounting for these theoretical systematic effects in the galaxy power spectrum is crucial to obtain unbiased inference about $\fnl$ \citep[see, e.g.,][]{riquelme2022primordial}.
 
 
\subsubsection{Angular power spectrum}
\mr{Would non-linear P(k) matter?} The relationship between the linear matter power spectrum $P(k)$ and the projected angular power spectrum of galaxies is expressed by the following equation:
\begin{equation}\label{eq:cell}
C_{\ell} = \frac{2}{\pi}\int_{0}^{\infty}\frac{dk}{k}k^{3}P(k)|\Delta_{\ell}(k)|^{2} + N_{\rm shot},
\end{equation}
where $N_{\rm shot}$ is a scale-independent shot noise term. The projection kernel $\Delta_{\ell}(k) = \Delta^{\rm g}_{\ell}(k) + \Delta^{\rm RSD}_{\ell}(k)$ includes redshift space distortions and determines the contribution of each wavenumber $k$ to the galaxy power spectrum on mode $\ell$. For more details, refer to \cite{Padmanabhan2007}. The FFTLog\footnote{\href{https://github.com/xfangcosmo/FFTLog-and-beyond}{github.com/xfangcosmo/FFTLog-and-beyond}} algorithm and its extension as implemented in \cite{fang2020beyond} are employed to calculate the integrals for the projection kernel $\Delta_{\ell}(k)$, which includes the $l^{\rm th}$ order spherical Bessel functions, $ j_{\ell}(kr)$, and its second derivatives,
\begin{align}
    \Delta^{\rm g}_{\ell}(k) &= \int \frac{dr}{r} r (b+\Delta b) D(r) \frac{dN}{dr} j_{\ell}(kr),\\
    \Delta^{\rm RSD}_{\ell}(k) &= - \int \frac{dr}{r} r f(r) D(r) \frac{dN}{dr} j^{\prime\prime}_{\ell}(kr),
\end{align}
where $b$ is the linear bias (dashed curve in Figure \ref{fig:nz}), $D$ represents the linear growth factor normalized as $D(z=0)=1$, $f(r)$ is the growth rate, and $dN/dr$ is the redshift distribution of galaxies normalized to unity and described in terms of comoving distance\footnote{$dN/dr = (dN/dz)*(dz/dr) \propto (dN/dz)*H(z)$} (solid curve in Figure \ref{fig:nz}). The PNG-induced scale-dependent shift is given by \citep[see, also,][]{slosar2008constraints}
\begin{equation}
\Delta b = b_{\phi}(z) \fnl \frac{3 \Omega_{m} H^{2}_{0}}{2 k^{2}T(k)D(z) c^{2}} \frac{g(\infty)}{g(0)},
\label{eq:scaledepbias}
\end{equation}
where $\Omega_{m}$ is the matter density, $H_{0}$ is the Hubble constant\footnote{$H_{0}=100~({\rm km}~{\rm s}^{-1})/(h^{-1}{\rm Mpc})$ and $k$ is in unit of $h {\rm Mpc}^{-1}$}, $T(k)$ is the transfer function, and $g(\infty)/g(0) \sim 1.3$ with $g(z)\equiv (1+z) D(z)$ is the growth suppression due to non-zero $\Lambda$ because of our normalization of $D$ \citep[see, e.g.,][]{2010JCAP...07..013R, 2019MNRAS.485.4160M}. \mr{Assuming that only mass determines the halo occupation function, $b_{\phi} = 2 \delta_{c}(b - p)$, where $p=1$ and $\delta_{c}= 1.686$ is the critical density for spherical collapse \citep{fillmore1984self}.} This paper is focused on how a careful assessment of imaging systematic effects, or lack thereof, can bias our PNG constraints. Therefore, we choose $p=1$ for our sample of DESI LRG targets \citep[see, also,][]{slosar2008constraints,2010JCAP...07..013R,2013MNRAS.428.1116R}, and do not marginalize over $p$ to avoid projection effects \citep{2022JCAP...11..013B}. \mr{Santi: p=1 for galaxies assumes this on top of assuming that halos follow a universal halo mass function (only depending on peak height)}.

\subsubsection{Survey geometry and integral constraint}
\mr{We employ a technique similar to the one proposed by \cite{chon2004fast} to account for the impact of the survey geometry on the theoretical power spectrum.} The ensemble average for the partial sky power spectrum is related to that of the full sky power spectrum via a mode-mode coupling matrix, ${\rm M}_{\ell \ell^{\prime}}$,
\begin{equation}\label{eq:mixm}
    <\tilde{C}_{\ell}> = \sum_{\ell^{\prime}} {\rm M}_{\ell \ell^{\prime}}<C_{\ell^{\prime}}>.
\end{equation}
\mr{We convert this convolution in the spherical harmonic space into a multiplication in the correlation function space. Specifically, we first transform the theory power spectrum (Equation \ref{eq:cell}) to the correlation function, $\hat{\omega}^{\rm model}$. Then, we estimate the survey mask correlation function, $\hat{\omega}^{\rm window}$, and obtain the pseudo-power spectrum,}
\begin{align}
    \tilde{C}^{\rm model}_{\ell} &= 2\pi \int \hat{\omega}^{\rm model}\hat{\omega}^{\rm window}~P_{\ell}(\cos \theta) d\theta.
\end{align}

 \begin{figure}
\centering
\includegraphics[width=0.45\textwidth]{model_mock.pdf}
\caption{The mean power spectrum from the $\fnl=0$ mocks (no contamination) and best-fitting theoretical prediction after accounting for the survey geometry and integral constraint effects. The dark and light shades represent $1\sigma$ error on the mean and one realization, respectively. Bottom panel shows the residual power spectrum relative to the mean power spectrum. No imaging systematic cleaning is applied to these mocks.}\label{fig:model_mock}
\end{figure}

The integral constraint is another systematic effect which is induced since the mean galaxy density is estimated from the observed galaxy density. The estimate of the mean density is biased by the limited sky coverage. This issue was first raised in \cite{peacock1991large}. To account for the integral constraint, the survey mask power spectrum is used to introduce a scale-dependent correction factor that needs to be subtracted from the power spectrum. Finally, the pseudo power spectrum with the integral constraint correction is obtained as
\begin{equation}
     \tilde{C}^{\rm model, IC}_{\ell} = \tilde{C}^{\rm model}_{\ell} - \tilde{C}^{\rm model}_{\ell=0} \left(\frac{\tilde{C}^{\rm window}_{\ell}}{\tilde{C}^{\rm window}_{\ell=0}}\right),
\end{equation}
\mr{where $\tilde{C}^{\rm window}$ is the spherical harmonic transform of $\hat{\omega}^{\rm window}$.}

The lognormal simulations are used to validate our survey window and integral constraint correction. Figure \ref{fig:model_mock} shows the mean power spectrum of the $\fnl=0$ simulations (dashed) and the best-fitting theory prediction before and after accounting for the survey mask and integral constraint. The simulations are neither contaminated nor mitigated. The light and dark shades represent the 68\% estimated error on the mean and one single realization, respectively. The DESI mask, which covers around $40\%$ of the sky, is applied to the simulations. We find that the survey window effect affects the clustering power on $\ell < 200$ and the integral constraint modulates the clustering power on $\ell < 6$.

\subsection{Parameter estimation}

\begin{figure*}
\centering
\includegraphics[width=0.85\textwidth]{hist_cl.pdf}
\caption{The distribution of the first bin power spectra and its log transformation from the simulations with $\fnl=0$ (left) and $76.92$ (right). The log transformation alleviates the asymmetry in the distributions.}\label{fig:histcell}
\end{figure*}

Our parameter inference uses standard MCMC sampling. A constant clustering amplitude is assumed for the linear bias of our DESI LRG targets, $b(z) = b/D(z)$. In MCMC, we allow $\fnl$, $N_{\rm shot}$, and $b$ to vary, while all other cosmological parameters are fixed at the fiducial values (see \S \ref{ssec:mocks}). \mr{The galaxy power spectrum is divided into a discrete set of bandpower bins with $\Delta\ell=2$ between $\ell=2$ and $20$ and $\Delta \ell=10$ from $\ell=20$ to $300$. Each clustering mode is weighted by $2\ell+1$ when averaging over the modes in each bin.}

With the lognormal simulations, we find that the distribution of the power spectrum at the lowest bin, $2\leq \ell < 4$, is asymmetric and its standard deviation varies significantly from the simulations with $\fnl=0$ to those with $76.9$ (Figure \ref{fig:histcell}). Therefore, we attempt to fit the log transformed power spectrum, $\log C_{\ell}$, to make our $\fnl$ constraints less sensitive to the choice of covariance matrix. The parameter $\fnl$ is constrained by minimizing a negative likelihood defined as,
\begin{equation}
-2\ln\mathcal{L} = (\log \tilde{C}(\Theta)-\log \tilde{C})^{\dagger} \mathbb{C}^{-1} (\log \tilde{C}(\Theta)-\log \tilde{C}),
\end{equation}
where $\Theta$ represents a container for the parameters $\fnl$, $b$, and $N_{\rm shot}$; $\tilde{C}(\Theta)$ is the (binned) expected pseudo-power spectrum; $\tilde{C}$ is the (binned) measured pseudo-power spectrum; and $\mathbb{C}$ is the covariance matrix constructed from the lognormal simulations. \mr{Flat priors are implemented for all parameters: $\fnl \in [-1000, 1000]$, $N_{\rm shot} \in [-0.001, 0.001]$, and $b \in [0, 5]$.} \mr{Anna: Some readers might wonder if the covariance estimated from the lognormal mocks is accurate enough. Could you add maybe some lines justifying this?}


\subsection{Characterization of remaining systematics}
\label{ssec:characterization}
\mr{Santi: I feel like this section would make more sense in "2.Data"}
In the absence of systematic effects, a) the mean galaxy density should be uniform across the footprint within the statistical fluctuations regardless of imaging conditions and b) the cross power spectrum between the galaxy density and the imaging systematic maps should be consistent with zero within the statistical fluctuations. In the following, two statistical tests are implemented and applied to quantify remaining systematic effects in our sample \cite[see, also,][]{rezaie2021primordial}.

\begin{figure*}
\centering
\includegraphics[width=0.95\textwidth]{clx_mocks.pdf}
\caption{\mr{The normalized cross power spectra between the DESI LRG targets and imaging systematic maps}: Galactic extinction (EBV), stellar density (nStar), depth in \textit{grzw1} (depth$_{grzw1}$), and seeing in \textit{grz} (psfsize$_{grz}$). The black curves display the cross spectra before imaging systematic correction. The red, blue and orange curves represent the results after applying the imaging weights from the linear models trained with \textit{eight maps}, \textit{two maps}, and \textit{three maps}. The green and pink curves display the results after applying the imaging weights from the non-linear models trained with \textit{three maps} and \textit{four maps}. The dark and light shades represent the $97.5$ percentile from cross correlating the imaging systematic maps and the $\fnl=0$ and $76.9$ lognormal density fields, respectively.}\label{fig:clxmock}
\end{figure*}

\begin{figure*}
\centering
\includegraphics[width=0.95\textwidth]{nbar_mocks.pdf}
\caption{The mean density contrast of the DESI LRG targets as a function of the imaging systematic maps: Galactic extinction (EBV), stellar density (nStar), depth in \textit{grzw1} (depth$_{grzw1}$), and seeing in \textit{grz} (psfsize$_{grz}$). The black curves display the results before imaging systematic correction. The red, blue and orange curves represent the relationships after applying the imaging weights from the linear models trained with \textit{eight maps}, \textit{two maps}, and \textit{three maps}. The green and pink curves display the results after applying the imaging weights from the non-linear models trained with \textit{three maps} and \textit{four maps}. The dark and light shades represent the $68\%$ dispersion of 1000 lognormal mocks with $\fnl=0$ and $76.92$, respectively.}\label{fig:nbarmock}
\end{figure*}


\begin{figure}
\raggedleft
\includegraphics[width=0.45\textwidth]{chi2test.pdf}
\includegraphics[width=0.44\textwidth]{chi2test2.pdf}
\caption{The remaining systematic error $\chi^{2}$ from the galaxy-imaging cross power spectrum (top) and the mean galaxy density contrast (bottom). The values observed in the DESI LRG targets before and after linear and non-linear treatments are quoted, and the histograms are constructed from 1000 realizations of clean lognormal mocks with $\fnl=0$ and $76.92$. \mr{Alex: can you show the DR9 chi2s as vertical lines rather than as text?}\mr{Ashley: I think you could add one vertical dashed and/or dotted line to show the result for the Nonlinear 3 maps, which is what we adopt as the fiducial case.}}\label{fig:chi2test}
\end{figure}


\subsubsection{Cross power spectrum}

We characterize the cross correlations between the galaxy density and imaging \mr{systematic} maps by \bbk{[Could we call them something else here (maybe contaminant maps)? I first got confused and thought you were cross-correlating redshift slices in your imaging survey, but maybe our readers will read more carefully than me]}
\begin{equation}
\tilde{C}_{X, \ell} = [\tilde{C}_{x_{1}, \ell}, \tilde{C}_{x_{2}, \ell}, \tilde{C}_{x_{3}, \ell}, ..., \tilde{C}_{x_{9}, \ell}],
\end{equation}
where $\tilde{C}_{x_{i}, \ell}$ represents \mr{the normalized cross power spectrum, determined from the the square of} the cross power spectrum between the galaxy density and $i^{\rm th}$ imaging map, $x_{i}$, divided by the auto power spectrum of $x_{i}$:
\begin{equation}
\tilde{C}_{x_{i}, \ell} = \frac{(\tilde{C}_{gx_{i}, \ell})^{2}}{\tilde{C}_{x_{i}x_{i},\ell}}.
\end{equation}
\mr{This normalization of $\tilde{C}_{x_{i}}$ estimates the contribution of systematics up to the first order to the galaxy power spectrum}. Then, the $\chi^{2}$ value for the cross power spectra is calculated via,
\begin{equation}
\chi^{2} = \tilde{C}^{T}_{X, \ell} \mathbb{C}_{X}^{-1} \tilde{C}_{X, \ell},
\end{equation}
where the covariance matrix $\mathbb{C}_{X} = < \tilde{C}_{X, \ell} \tilde{C}_{X, \ell'} >$ is constructed from the lognormal mocks. These $\chi^{2}$ values are measured for every clean mock realization with the \textit{leave-one-out} technique and compared to the values observed in the LRG sample with various imaging systematic corrections. \bbk{[Can you elaborate a more what you mean by that?]} \mr{Specifically, we use 999 realizations to estimate a covariance matrix and then apply the covariance matrix from the 999 realizations to measure the $\chi^{2}$ for the one remaining realization. This process is repeated for all 1000 realizations to construct a histogram for $\chi^{2}$.} We only include the bandpower bins from $\ell=2$ to $20$ with $\Delta\ell=2$, and test for the robustness with higher $\ell$ modes in Appendix \ref{sec:scalesys}. 

Figure \ref{fig:clxmock} shows $\tilde{C}_{X}$ from the DESI LRG targets before and after applying various corrections for imaging systematics. The dark and light shades show the 97.5$^{\rm th}$ percentile from the $\fnl=0$ and $76.9$ mocks, respectively. \mr{Without imaging weights, the LRG sample has the highest cross-correlations against extinction, stellar density, and depth in z (solid black curve). There is less significant correlations against depth in the g and r bands, and psfsize in the z band, which could be driven because of the inner correlations between the imaging systematic maps}. First, we consider cleaning the sample with the linear model using two maps (extinction and depth in r) as identified from the Pearson correlation. With linear two maps (dot-dashed blue curve), most of the cross power signals are reduced below statistical uncertainties, especially against extinction, stellar density, and depth. However, the cross power spectra against psfsize in r and z increases slightly on $6<\ell<20$ and $6<\ell<14$, respectively. Very likely, large-scale cross correlations ($\ell < 6$) are reduced using extinction and depth in the z-band, but there are some residual cross correlations on smaller scales ($\ell > 6$) which cannot be mitigated with our set of two maps. The linear three maps (dotted orange curve) approach alleviates the cross power spectrum against psftsize in r, but not psfsize in z. On the other hand, non-linear three maps can reduce the cross correlations against both the r and z-band psfsize maps, which indicates the benefit of using a non-linear approach. For benchmark, we also show the normalized cross spectra after cleaning the LRG sample with linear eight maps and non-linear four maps. 

\subsubsection{Mean density contrast}
We calculate the histogram of the mean density contrast relative to the $j^{\rm th}$ imaging property, $x_{j}$:
\begin{equation}
\delta_{x_{j}} = ({\overline{\rho}})^{-1} \frac{\sum_{i} \rho_{i} W_{i}}{\sum_{i} W_{i}},
\end{equation}
\mr{where $\overline{\rho}$ is the global mean galaxy density, $W_{i}$ is the survey window in pixel $i$, and the summations over $i$ are evaluated from the pixels in every bin of $x_{j}$. We compute the histograms against all other imaging properties (see Figure \ref{fig:ng}). We use a set of eight equal-width bins for every imaging map, which results in a total of 72 bins. Then, we construct the total mean density contract as,}
\begin{equation}
\delta_{X} = [\delta_{x_{1}}, \delta_{x_{2}}, \delta_{x_{3}}, ..., \delta_{x_{9}}],
\end{equation}
and the total residual error as,
\begin{equation}
\chi^{2} = \delta_{X}^{T} \mathbb{C_{\delta}}^{-1} \delta_{X},
\end{equation}
where the covariance matrix $\mathbb{C}_{\delta} = < \delta_{X} \delta_{X}>$ is constructed from the lognormal mocks. Figure \ref{fig:nbarmock} shows the mean density contrast against the imaging properties for the DESI LRG targets. The dark and light shades represent the $1\sigma$ level fluctuations observed in 1000 lognormal density fields respectively with $\fnl=0$ and $76.92$. The DESI LRG targets before treatment (solid curve) exhibits a strong trend around $10\%$ against the z-band depth which is consistent with the cross power spectrum. Additionally, there are significant spurious trends against extinction and stellar density at about $5-6\%$. The linear approach is able to mitigate most of the systematic fluctuations with only extinction and depth in the z-band as input; however,  a new trend appears against the r-band psfsize map with the \textit{linear two maps} approach (dot-dashed blue curve), which is indicative of the psfsize-related systematics in our sample. This finding is in agreement with the cross power spectrum. We re-train the linear model with three maps, but we still observe around $2\%$ residual spurious fluctuations in the low ends of depth and psfsize, in the z band, which implies non-linear systematic effects exist. We find that the imaging weights from the non-linear model trained with the three identified maps (or four maps including the stellar density) is capable of reducing the fluctuations below $2\%$. \mr{Even with the non-linear three maps, we have about $1\%$ remaining systematic fluctuations against the z-band psfsize. We use the $\chi^{2}$ statistics to assess how significant these fluctuations are in comparison to the clean mocks.}

\mr{Figure \ref{fig:chi2test} shows the $\chi^{2}$ histograms from the normalized cross spectrum (top) and mean density contrast (bottom) statistics which are obtained from the $\fnl=0$ and $76.9$ lognormal mocks, respectively with dashed and solid lines. No mitigation is applied to these mocks, and thus the $\chi^{2}$ values are expected to be unbiased. The $\chi^{2}$ values observed in the DESI LRG targets are quoted for comparison, and the significance for each measurement is characterized by \textit{p-value} with respect to the $\fnl=0$ mocks without any mitigation applied to them}. Before cleaning, our LRG sample has a cross power spectrum $\chi^{2}$ error of $20014.8$. After correction with the linear two maps approach, the cross spectrum $\chi^{2}$ is reduced to $375.1$ with p-value $=0.002$. Adding the r-band psfsize, the linear model reduces the $\chi^{2}$ down to $195.9$ with p-value $=0.044$; we can reject the null hypothesis that our sample with the linear three maps is properly cleaned at $95\%$ confidence. Even though training the linear model with all imaging systematic maps as input gives the lowest cross spectrum $\chi^{2}$ of $129.2$ (and p-value $=0.239$), it potentially makes the analysis more prone to over-fitting and regressing out the true clustering signal, given the inner correlations among the imaging properties (Figure \ref{fig:pcc}). As an alternative, we apply the imaging weights from the non-linear method with the extinction, z-band depth, and r-band psfsize maps (\textit{non-linear three maps}). The cross power spectrum $\chi^{2}$ is reduced to $79.3$ with p-value $=0.594$.  Adding the stellar density map reduces the cross power spectrum $\chi^{2}$ error to $70.9$ (p-value $=0.687$). Our cross power spectrum diagnostic supports the idea that a non-linear cleaning approach is needed to properly regress out the remaining spurious fluctuations. We investigate the test with the cross power spectrum up to higher multipoles but find no evidence of remaining systematic errors (see Appendix \ref{sec:scalesys}). 

Figure \ref{fig:chi2test} (bottom) shows the mean density $\chi^{2}$ observed in the mocks with or without $\fnl$. We find consistent results regardless of the underlying $\fnl$, which supports that our diagnostic is not sensitive to the fiducial cosmology. The values measured in the DESI LRG targets before and after applying imaging weights are quoted for comparison. The \textit{linear two maps} weights reduce the $\chi^{2}$ value from $679.8$ (before correction) to $178.8$. The p-value $=0$ indicates severe remaining systematic effects. Adding the r-band psfsize does not reduce the p-value enough (e.g., greater than $0.05$) even though the cleaning method yields a lower $\chi^{2}=130$. Training the linear model with all imaging systematic maps returns a more reasonable $\chi^{2}=90$ and p-value of $0.084$. However, regression with all imaging systematic maps as input can lead to the removal of the true clustering signal. With the imaging weights from the \textit{non-linear three maps} approach, we obtain a $\chi^{2}$ value of $74.3$ with p-value $=0.392$. Re-training the non-linear approach while adding the stellar density map (\textit{non-linear four maps}) yields minor improvement: $\chi^{2}=73.2$ and p-value $=0.422$.  This indicates that the stellar density trend in the mean LRG density can be explained via the extinction map. \mr{Ashley: Add something at the end to clarify the reason that the extinction map can cause the stellar density trend is that they are strongly correlated.}

\mr{Ashley: I would maybe add a new subsubsection (but at least a new paragraph) to summarize the results of the subsection, as they are quite important. This is also where I would discuss the tests that were done prior to looking at the data Cell. The discussion would basically provide a strong justification for the 3 map case being the fiducial case and point out that this was decided prior to seeing the data results. (Some of the words you wrote at the beginning of section 3.5 would go instead in this summary.)}




\subsection{Calibration of mitigation bias}\label{ssec:calibration}

\mr{Santi: How all of this was computed did not seem very clear to me. It seems that this part is what later determines that fnl!=0 from the data. So I think it would be worth explaining it more.} \mr{Ashley: I would find a way to *strongly* emphasize that the result does not depend at all on how the mocks were contaminated. Perhaps just a concluding sentence or two in its own paragraph along the lines}
The template-based mitigation of imaging systematics removes some of the true clustering signal, and the amount of the removed signal increases as more maps are fed to the regression. Supported by our remaining systematic test, \textit{non-linear three maps} is therefore chosen as our default approach and is applied to the mocks (with and without contaminations) to calibrate the $\fnl$ biases induced by over-correction. Below we describe an approach for the calibration and de-biasing of our $\fnl$ constraints. 

\begin{figure}
\centering
\includegraphics[width=0.45\textwidth]{figures/fnlbias}
\caption{The \textit{No mitigated, clean} vs \textit{mitigated} $\fnl$ values from the $\fnl=0$ and $76.9$ mocks. The solid (dashed) lines represent the best-fitting estimates from fitting the mean power spectrum of the clean (contaminated) mocks. The scatter points show the best-fitting estimates from fitting the individual spectra for the clean mocks.}\label{fig:fnlbias}
\end{figure}



\begin{table}
\begin{center}
\caption{Linear parameters employed to de-bias the $\fnl$ constraints to account for the over-correction issue. \mr{Ashley: You could also add the m1/m2 to the table for the clean mock case?}}\label{tab:debiasparams}
\begin{tabular}{lcc}
\hline
\hline
\textbf{Cleaning Method} & $m_{1}$ & $m_{2}$ \\
\hline
Non-linear Three Maps & 1.17 & 13.95 \\
Non-linear Four Maps & 1.32 & 26.97 \\
Non-linear Nine Maps & 2.35 & 63.5\\
\hline
\end{tabular}
\end{center}
\end{table}


To calibrate for over-correction, we utilize our series of lognormal density fields with and without PNG, with and without systematic effects. The contamination model is based on the linear multivariate approach with the extinction, z-band depth, and r-band psfsize maps as input and parameters drawn from the likelihood constrained by the DESI LRG targets. The idea is to simulate systematic effects that reflect spurious fluctuations as  realistic as the DESI LRG targets. For correction, the neural network model is trained and applied to the simulations with various sets of imaging systematic maps as input. Particularly, we consider \textit{non-linear three maps}, \textit{non-linear four maps}, and \textit{non-linear nine maps}. We fit both the mean power spectrum and each individual power spectrum of 1000 realizations. The best-fitting estimates from the mocks without systematics (and no mitigation applied) are considered as the true $\fnl$ values and the estimates from the mocks (with the mitigation procedure applied) are considered as the measured $\fnl$ values. Figure \ref{fig:fnlbias} shows the best-fitting estimates of $\fnl$ before mitigation vs the best-fitting $\fnl$ estimates after mitigation from the mean power spectrum (solid lines) and individual spectra (points) for the mocks without systematics. The dashed lines represent the best-fitting estimates from the contaminated realizations. The best-fitting estimates for the individual contaminated mocks are not shown for visual clarity. 

Then, a pair of linear parameters are found to transform the $\fnl$ values after mitigation to those before mitigation, $f_{\rm NL, no~mitigation, clean} = m_{1} f_{\rm NL, mitigated} + m_{2}$. These $m_{1}$ and $m_{2}$ coefficients for non-linear three, four, and nine maps are summarized in Table \ref{tab:debiasparams}. The uncertainty in $\fnl$ increases by $m_{1}-1$. We find that $m_{1}-1$ determines the added uncertainty in the $\fnl$ constraints, once the correction coefficients are applied. We expect this effect to be maximum for \textit{non-linear nine maps} and minimum for \textit{non-linear three maps}, as the effect of over-correction is more significant when the number of imaging systematics maps as input to the neural network increases. 

\mr{Emphasize that the mitigation bias or over-correction effect is the same regardless of the mocks being contaminated or not.} \mr{Ashley: It is important to note that we find nearly identical results for the mitigation bias whether or not the mocks have any contamination. This can be seen by observing the solid and dashed curves displayed on Fig. 9}