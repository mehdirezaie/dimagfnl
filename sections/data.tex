\section{Data}
\label{sec:data}
Luminous red galaxies (LRGs) are massive galaxies that occupy massive halos, lack active star formation, and are one of the highly biased tracers of dark matter gravitational field. A distinct break around 4000 \AA~in the spectrum of LRGs is often utilized to determine their redshifts accurately. LRGs are widely targeted in previous galaxy redshift surveys \citep[see, e.g.,][]{eisenstein2001spectroscopic, prakash2016sdss}, and their clustering and redshift properties are well studied \citep[see, e.g.,][]{ross2020MNRAS.498.2354R, gilmarin2020MNRAS.498.2492G, bautista2021MNRAS.500..736B, chapman2022MNRAS.516..617C}. DESI is designed to collect spectra of millions of LRGs covering the redshift range of $0.4<z<1.0$ throughout its five-year mission \citep{aghamousa2016desi}. DESI spectroscopy selects its targets from photometry of three ground-based surveys that observed the sky in the optical $g$, $r$, and $z$ bands between 2014 and 2019: the Mayall $z$-band Legacy Survey using the Mayall telescope at Kitt Peak \citep[MzLS;][]{dey2018overview}, the Beijing–Arizona Sky Survey using the Bok telescope at Kitt Peak \citep[BASS;][]{zou2017project}, and the Dark Energy Camera Legacy Survey on the Blanco 4m telescope \citep[DECaLS][]{flaugher2015dark}. The BASS and MzLS surveys observed the same footprint in the North Galactic Cap (NGC) while the DECaLS observed both caps around the galactic plane; the BASS+MzLS footprint is separated from the DECaLS NGC at DEC $> 32.375$ degrees, although there is an overlap between the two regions for calibration purposes \citep{dey2018overview}. Additionally, the DECaLS program integrates observations executed from the same instrument under the Dark Energy Survey \citep{abbott2016dark}, which constitute about $1130 \deg^{2}$ of the South Galactic Cap (SGC) footprint. The DESI imaging catalogs also integrate the $3.4$ (W1) and $4.6$ $\mu m$ (W2) infrared photometry from the Wide-Field Infrared Explorer \citep[WISE;][]{wise2010AJ....140.1868W, meisner2018RNAAS...2....1M}. 

\subsection{DESI imaging LRG sample}
We work with the photometric LRGs selected from the DESI Legacy Imaging Surveys Data Release 9 \citep[DR9;][]{dey2018overview} using the selection criteria designed for the DESI 1\% survey \mr{(CITE)}, described as the SV3 selection in more detail in \cite{zhou2022target}. The color-magnitude selection cuts are defined in the $g$, $r$, $z$ bands in the optical and $W1$ band in the infrared, as summarized in Tab. \ref{tab:ts}. The selection cuts are developed differently for each imaging survey to reach an almost uniform target surface density despite different survey efficiency and photometric calibration between DECaLS and BASS+MzLS. The implementation of these selection cuts in the DESI data processing pipeline is explained in \cite{myers2022}. Fig. \ref{fig:nz} shows the redshift distribution of the DR9 LRGs (solid black), inferred from the spectroscopic DESI Survey Validation data \mr{(CITE)}, and the evolution of halo bias (red dashed), adopted from \cite{zhou2021clustering}.

\begin{figure}
 \centering
 \includegraphics[width=0.5\textwidth]{figures/nz_lrg.pdf}
 \caption{Redshift distribution (solid) and bias evolution (dashed) of the DR9 LRG sample \citep{zhou2021clustering, zhou2022target}. The redshift distribution is inferred from spectroscopy in the DESI Survey Validation phase, and the model for bias assumes a constant clustering amplitude.}
 \label{fig:nz}
\end{figure}

\begin{table*}
 \caption{Selection criteria for the DESI-like LRG targets \citep{zhou2022target}. Magnitudes are corrected for MW extinction. $z_{\rm fiber}$ represents the z-band fiber magnitude which corresponds to the expected flux within a DESI fiber.}
 \label{tab:ts}
 \centerline{%
 \begin{tabular}{lll}
 \hline
 \hline
  \textbf{Footprint} & \textbf{Criterion} &\textbf{Description}\\
  \hline
  \hline  
 & $z_{\rm fiber} < 21.7$ & Faint limit \\
   DECaLS & $z - W1 > 0.8 \times (r - z) - 0.6$ & Stellar rejection \\
  & $[(g-r >1.3)~{\rm AND}~((g-r) > -1.55*(r-W1) + 3.13)]~{\rm OR}~(r -W 1 > 1.8)$ & Remove low-z galaxies \\
  & $[(r-W1 > (W1 - 17.26)*1.8)~{\rm AND}~(r - W1 > W1 - 16.36)]~{\rm OR}~(r-W1 > 3.29)$ & Luminosity cut \\ 
 \hline
 & $z_{\rm fiber} < 21.71$ & Faint limit \\
 BASS+MzLS & $z - W1 > 0.8 \times (r - z) - 0.6$ & Stellar rejection \\
 & $[(g-r >1.34)~{\rm AND}~((g-r) > -1.55*(r-W1) + 3.23)]~{\rm OR}~(r -W 1 > 1.8)$ & Remove low-z galaxies \\
 & $[(r-W1 > (W1 - 17.24)*1.83)~{\rm AND}~(r - W1 > W1 - 16.33)]~{\rm OR}~(r-W1 > 3.39)$ & Luminosity cut \\ 
  \hline
  \end{tabular}
  }
\end{table*}

\begin{figure*}
 \centering
 \includegraphics[width=\textwidth]{figures/dr9data.pdf}
 \caption{DESI Legacy Imaging Survey Data Release 9 Luminous Red Galaxies and imaging properties \citep{dey2018overview}. Top: Observed target density field in deg$^{-2}$. Spurious disconnected islands from the DECaLS North footprint at Declination below $-11$ and parts of the DECaLS South with Declination below $-30$ are rejected from the DR9 sample due to potential calibration issues. Bottom: Mollweide projections of the DR9 catalog imaging properties (survey depth and astronomical seeing/psfsize) and MW foregrounds (extinction and local stellar density) in celestial coordinates.}
 \label{fig:ng}
\end{figure*}

DESI-like LRGs are selected brighter than the imaging survey depth limits; therefore, the DR9 LRG density field is nearly homogenous, unlike the other DESI tracers. To further reduce stellar contamination, the LRG sample is masked rigorously for foreground bright stars, galaxies, and clusters of galaxies\footnote{See the maskbits at \url{https://www.legacysurvey.org/dr9/bitmasks/}}. Then, the sample is binned into \textsc{healpix} \citep{gorski2005healpix} at $\textsc{nside}=256$ to construct the 2D density map with an average surface density of $800$ deg$^{-2}$ with sky coverage around $14000$ square degrees. We correct for the pixel incompleteness and lost areas in the density field of LRGs using a catalog of random points, hereafter referred to as randoms, uniformly scattered over the footprint with the same cuts and masks applied. Fig. \ref{fig:ng} (top) shows the observed density field of the DR9 LRGs in deg$^{-2}$ before applying any correction weights to account for imaging systematic effects. The DR9 LRG density exhibits large-scale spurious fluctuations, which are unlikely to be of cosmological origin. Specifically, the SGC footprint exhibits some systematic under-density while there is some systematic over-density near the survey boundaries in the NGC. 

\subsubsection{Correlation coefficient analysis}
We study the correlation between the DR9 LRG sample and imaging properties as potential sources of systematic error, mapped into \textsc{HEALPix} at the same \textsc{nside}. Following \cite{zhou2022target}, imaging properties investigated in this work are local stellar density constructed from point-like sources with a g-band magnitude in the range $12 \leq g < 17$ from the Gaia DR2 \citep[see,][]{gaiadr2, myers2022}; Galactic extinction E[B-V] from \cite{schlegel1998maps}; and survey-related imaging properties include survey depth (galaxy depth in the $g$, $r$, and $z$ bands and PSF depth in W1) and astronomical seeing (psfsize) in the $g$, $r$, and $z$ bands. Templates for these survey-related imaging properties are produced by making the histogram of randoms, which are painted with imaging properties, in \textsc{HEALPix} and averaging over the attributes of randoms in each pixel. 

Fig. \ref{fig:ng} (bottom) illustrates the imaging templates investigated as potential sources of systematic error. Each map shows its own characteristic large-scale spurious fluctuations. For instance, the under-dense part of the DR9 LRG sample in the SGC can be associated with survey depth, while the over-density in the NGC can be linked to the extinction map. We reject some parts of the DR9 sample to minimize the potential for photometric calibration systematics. There are some disconnected islands, hereafter referred to as \textit{spurious islands}, in the DECaLS North region at Dec $< -11$. Additionally, some parts of the DECaLS South footprint with Dec $< -30$ are removed from the sample, because a different catalog of standard stars is employed to calibrate images below that region. We discuss how these quality cuts influence $\fnl$ constraints from the DR9 LRG sample in Section \ref{sec:results}. 

\begin{figure}
 \includegraphics[width=0.45\textwidth]{figures/pcc.pdf} 
 \includegraphics[width=0.45\textwidth]{figures/pccx.pdf}  
 \caption{Top: Pearson correlation coefficients between galaxy density and imaging properties in the three imaging regions. Solid curves represent the $95\%$ spread of correlation coefficients observed in 100 randomly selected lognormal mock density realizations. Bottom: Pearson correlation matrix from imaging properties for the full DESI footprint.}
 \label{fig:pcc}
\end{figure}



Fig. \ref{fig:pcc} shows the Pearson correlation coefficient between the DR9 LRG density and DESI imaging properties for the three imaging surveys (DECaLS North, DECaLS South, and BASS+MzLS) in the top panel. The horizontal curves are constructed from lognormal simulations (see, subsection \ref{ssec:mocks}) to quantify the significance of correlations. Fig. \ref{fig:pcc} (bottom) shows the correlation matrix among imaging properties for the DESI footprint. There is a strong correlation between the LRG density and depth maps, and next correlated properties seem to be Galactic foregrounds. There is a small correlation between the LRG density and the W1 depth and psfsize properties. We observe a significant inner correlation among the imaging properties themselves, especially between the local stellar density and Milky Way extinction; also, the $r$-band and $g$-band survey properties are more correlated with each other than with the $z$-band. 

\subsubsection{Imaging weights}
We follow a template-based approach to derive a set of weights for removing the effects of imaging systematics. The weights are organized in \textsc{HEALPix} and the LRG density map is multiplied by these weight maps in an attempt to reduce the level of spurious fluctuations in the LRG data. We regress the DR9 density map against a set of imaging maps, referred to as templates or predictors. Because of the inner correlation amongst the maps, a few subsets of imaging maps are considered as well for a conservative cleaning approach. These subsets are selected to minimize the correlations among the predictors while having a maximum correlation with the observed LRG density map:
\begin{itemize}
\item Conservative I: Extinction, depth$_{z}$
\item Conservative II: Extinction, depth$_{z}$, psfsize$_{r}$
\item All Maps: Extinction, depth in $grz$ and $W1$, psfsize in $grz$
\end{itemize}
We also investigate whether including external maps for neutral hydrogen column density \citep{2016A&A...594A.116H} and photometric calibration (e.g., in the z band; \textit{CALIBZ}) could provide more insights on remaining systematic effects. The results are discussed in Section \ref{sec:results}.

Linear and nonlinear models (approximated using a neural network) are applied to assess the potential of nonlinear systematic error. Parameters of the models are estimated by optimizing the negative Poisson log-likelihood, $= \sum \lambda - \rho \log(\lambda)$, where the summation runs over pixels, $\rho$ is the observed galaxy density, and $\lambda$ is the predicted galaxy density given imaging properties \textbf{x} as input, $\lambda(\textbf{x}) = \log (1+e^{f(\textbf{x})})$. We investigate the use of linear multivariate and nonlinear regression to approximate $f$. For finding the parameters of the linear model, we perform a Monte Carlo Markov Chain (MCMC) search using the \textsc{emcee} package \citep{2013PASP..125..306F}. For the nonlinear approximation, we employ the implementation of artificial neural networks from \cite{rezaie2021primordial}; specifically, the nonlinear model is an ensemble of 20 neural network models. Each neural network consists of three hidden layers and 20 rectifier units on each layer. The rectifier is the identity function for positive input and zero for negative, and it introduces nonlinearities in the neural network. For the linear model, we use all data for computing the log of posterior during MCMC while for the nonlinear approach, we use $60\%$ of data for training, $20\%$ for validation, and $20\%$ for testing; the training-validation-testing split is applied to minimize the chance of over-fitting by the nonlinear model. We test the nonlinear model on the entire DR9 data by changing the permutation of the training, validation, and testing sets. The neural networks are trained for up to 70 training epochs with \textsc{Adam} optimizer, which is a variant of stochastic gradient descent. We tune the initial learning rate by minimizing the loss on the validation set and adjust it to dynamically vary between two boundary values of $0.001$ and $0.1$ to avoid local minima during gradient descent. The best neural network model is then selected from the lowest prediction error when applied to the validation set. Finally, we apply the ensemble of 20 best-fit models to the test set and average over the predictions to construct the predicted galaxy density map in \textsc{HEALPix}. The predicted galaxy density is employed as weights to down-weight observed density map for removing imaging systematics. We train the linear and nonlinear models for different sets of input imaging maps. 

Upon inspecting the predicted density maps, we find that while most of the large-scale spurious fluctuations are explained by just the extinction map and depth in the z band, adding the r-band psfsize results in a finer structure in the predicted density map. We observe that using all imaging maps as input features for regression does not add much information. Comparing the linear to the nonlinear prediction for the same set of input maps, we find that the nonlinear approach yields finer structures due to higher flexibility. Overall, both models predict higher galaxy density near the boundaries where the DESI imaging surveys observed the high extinction regions of the Milky Way. These over-dense regions are likely contaminated artifacts entering the selection, e.g., stellar contaminats or other artifacts because of obscured photometry by MW extinction.

\subsection{Synthetic lognormal density fields}\label{ssec:mocks}
Density fluctuations of galaxies can be approximated with lognormal distributions on large scales \citep{coles1991}. Unlike N-body simulations, simulating lognormal density fields is not computationally intensive. We use \textsc{FLASK} \citep[Full-sky Lognormal Astro-fields Simulation Kit;][]{Xavier_2016} to generate ensembles of synthetic galaxy density fields that mimic the redshift and angular distributions of the DR9 LRG sample. We create 1000 realizations with $\fnl=0$ and $76.92$ using a redshift dependent bias $b(z)=1.43/D(z)$. We adapt the fiducial cosmology from a flat $\Lambda$CDM universe, including one massive neutrino with $m_{\nu}=0.06$ eV, and the rest of cosmological parameters is deducted from Planck 2018 \citep{aghanim2020planck},
\begin{equation*}
 h = 0.67, \Omega_{M}=0.31, \sigma_{8}=0.8, {\rm and}~ n_{s}=0.97.
\end{equation*}
The same fiducial cosmology is employed throughout the rest of the manuscript. We demonstrate later in Section \ref{sec:results} that our $\fnl$ constraints are quite robust against the choice of our fiducial cosmology.